\documentclass[
    type = bachelor,
    degree = academic,
    twoside,
    fontset = win
]
{njuthesis}

% njuthesis 参数设置文件 v1.4.0 2024-03-19

\njusetup[info]{
    title = {面向TLA\textsuperscript{+}规约的\\归纳不变式自动生成技术},
    title* = {Automatic Inductive Invariants Inference for  Specifications in TLA\textsuperscript{+} },
    author = {张继华},
    author* = {Jihua Zhang},
    keywords = {分布式协议, 形式化验证, 归纳不变式, TLA\textsuperscript{+}, 强化学习},
    keywords* = {Distributed Protocols, Formal Verification, Inductive Invariant, TLA\textsuperscript{+}, Reinforcement Learning},
    grade = {2020},
    student-id = {201250040},
    department = {软件学院},
    major = {软件工程},
    major* = {Software Engineering},
    supervisor = {魏恒峰,助理研究员},
    supervisor*= {Hengfeng Wei, Research Assistant},
    submit-date = {2024-05-27},
    % 提交日期
    supervisor-contact = {
        南京大学~
        江苏省南京市鼓楼区汉口路22号
    }
    % 导师联系方式
}

% bib 类用于参考文献设置
\njusetup[bib]{
    % style = numeric|author-year,
    % 参考文献样式
    % 默认为顺序编码制(numeric)
    % 可选著者-出版年制(author-year)
    %
    resource = {zhangjihua-thesis.bib},
    % 参考文献数据源
    % 需要带扩展名的完整文件名
    % 可使用逗号分隔多个文件
    % 此条等效于 \addbibresource 命令
    %
    % option = {
        % doi    = false,
        % isbn   = false,
        % url    = false,
        % eprint = false,
        % 关闭部分无用文献信息
        %
        % refsection = chapter,
        % 将参考文献表置于每章后
        %
        % gbnamefmt = lowercase
        % 使用仅首字母大写的姓名
    %   }
    % 额外的 biblatex 宏包选项
}

% image 类用于载入外置的图片
\njusetup[image]{
    % path = {{./figure/}{./image/}},
    % 图片搜索路径
    %
    nju-emblem = {nju-emblem},
    nju-name = {nju-name},
    % nju-emblem = {nju-emblem-purple},
    % nju-name = {nju-name-purple},
    % 替换为紫色版本
    % 这个选项只能填写一次
    % 切换时要注释掉上方的黑色版本
}

% abstract 类用于设置摘要样式
\njusetup[abstract]{
    toc-entry = true,
    % 摘要是否显示在目录条目中
    %
    % underline = false,
    % 研究生英文摘要页条目内容是否添加下划线
    %
    % title-style = strict|centered|natural
    % 研究生摘要标题样式,详见手册
}

% 目录自身是否显示在目录条目中
\njusetup{
    tableofcontents/toc-entry = false,
    % 关闭本项相当于同时关闭三个选项
    %
    % listoffigures/toc-entry   = false,
    % listoftables/toc-entry    = false
}

% 为目录中的章标题添加引导线
\njusetup[tableofcontents/dotline]{chapter}

% math 类用于设置数学符号样式,功能详见手册
\njusetup[math]{
    % style              = TeX|ISO|GB,
    % 整体风格,缺省值为国标(GB)
    % 相当于自动设置以下若干项
    %
    % integral           = upright|slanted,
    % integral-limits    = true|false,
    % less-than-or-equal = slanted|horizontal,
    % math-ellipsis      = centered|lower,
    % partial            = upright|italic,
    % real-part          = roman|fraktur,
    % vector             = boldfont|arrow,
    % uppercase-greek    = upright|italic
}

% theorem 类用于设置定理类环境样式,功能详见手册
\njusetup[theorem]{
    % define,
    % 默认创建内置的七种定理环境
    %
    % style         = remark,
    % header-font   = \sffamily \bfseries,
    % body-font     = \normalfont,
    % qed-symbol    = \ensuremath { \male },
    % counter       = section,
    % share-counter = true,
    % type          = {...}
    % 以上设置项在重新调用 theorem/define 后生效
}

% footnote 类用于设置脚注样式,功能详见手册
\njusetup[footnote]{
  % style = pifont|circled,
  % 使用圈码编号
  %
  % hang = false,
  % 不使用悬挂缩进
}

% 页眉页脚内容设置
\njusetup{
  % header/content = {
  %     {OR}{\thepage},{OL}{\rightmark},
  %     {EL}{\thepage},{ER}{\leftmark}
  %   },
  % 页眉设置,详见手册
  % 奇数页页眉:左侧章名,右侧页码
  % 偶数页页眉:左侧页码,右侧节名
  %
  % footer/content = {}
}

% 页眉页脚的字体样式
% \njusetformat{header}{\small\kaishu}
% \njusetformat{footer}{}

% 一些灵活调整
\njusetname{type}{本科毕业设计}                 % 我做的是毕业设计
% \njusetname{notation}{术语表}                   % 更改符号表名称
% \njusetlength{crulewd}{240pt}                   % 加长封面页下划线
% \njusetformat{tabular}{\zihao{-4}\bfseries}     % 修改表格环境的字号
% \EditInstance{nju}{u/cover/emblem-img}{align=l} % 左对齐的本科生封面校徽

\usepackage{listings} % 展示代码
\lstset{
    captionpos=b
}
\usepackage[commentsnumbered,ruled,linesnumbered,vlined]{algorithm2e}
\usepackage{subcaption} % 嵌套小幅图像,比 subfig 和 subfigure 更新更好
\usepackage{siunitx} % 标准单位符号
\usepackage{tabularx}
\usepackage{array}
\usepackage{booktabs}
\usepackage{appendix}
\usepackage{amsmath}
\usepackage{courier}
\usepackage{multirow}
\usepackage{pdfpages}

\lstset{
	frame=single,
        numbers=left,
	aboveskip=3mm,
	belowskip=3mm,
	showstringspaces=false,
	basicstyle=\footnotesize\ttfamily,
	columns=flexible,
	tabsize=2,
	keepspaces=true,
        breaklines=true,
        breakatwhitespace=true
}

\newcommand{\TLA}{TLA\textsuperscript{+}}
\newcommand{\rltla}{RL-TLA}

\begin{document}
% 封面
\maketitle
\includepdf[pages={1,{}}]{figures/decl-page.pdf}

\begin{abstract}
    分布式协议以及分布式系统,在当今的计算机世界不可或缺。自动化地对分布式协议验证其正确性是一个重要且困难的挑战。
    分布式协议的正确性表达为其定义的安全属性(safety property)在每个状态下都成立。
    对于复杂系统,我们无法像验证简单系统,通过遍历所有状态的方式来验证安全属性。
    已有的研究常常通过生成一个蕴含安全属性的归纳不变式的正确性的方式来验证安全属性的正确性,继而验证规约的正确性。

    自动化地生成分布式协议的归纳不变式是自动化验证分布式协议正确性的关键步骤。
    但是,自动化寻找归纳不变式是一个困难的问题,并且已有的研究主要基于的IVy进行实现,\TLA 语言领域的相关研究较少。
    本文将实现一个面向\TLA 规约的归纳不变式生成工具\rltla ,实现对以\TLA 语言描述的分布式协议规约的归纳不变式的自动化生成。
    与此同时,不同于以往通过优化遍历顺序的方式,本文将使用强化学习的方法,加速归纳不变式的推导过程。
    通过引入TLC或Apalache 模型检查器,实现对候选不变式和归纳不变式的验证。

    通过面向部分规约的实验,我们可以验证\rltla 的有效性,证明强化学习在归纳不变式生成中应用的可行性。

\end{abstract}
  
\begin{abstract*}
    Distributed protocols and distributed systems are indispensable in contemporary computer world. Automatical inference for the correctness of distributed protocols is an important and challenging task.
    The correctness of distributed protocols is expressed as the safety property defined by the protocol holds in every state.
    For complex systems, we cannot verify the safety property by traversing all states as for simple protocols.
    Existing research often generates the correctness of an inductive invariant that implies the safety property to infer the correctness of the safety property, and then infer the correctness of the protocol.
    
    Automatic generation of inductive invariants for distributed protocols is a key step in automatic verifying the correctness of distributed protocols.
    However, automatically finding inductive invariants is a difficult problem, and existing research is mainly based on IVy, with less research in the \TLA language domain.
    This paper will implement an inductive invariant generation tool \rltla for \TLA specifications, to automatically generate inductive invariants for distributed protocol specifications in \TLA language.
    Meanwhile, different from the previous way of optimizing the traversal order, this paper will use reinforcement learning to accelerate the derivation process of inductive invariants.
    By introducing TLC or Apalache, model checkers, the verification of candidate invariants and inductive invariants is realized.

    Through experiments, we can verify the effectiveness of \rltla and prove the feasibility of application of reinforcement learning in inductive invariant generation.

\end{abstract*}

% 目录
\tableofcontents

% 正文
\mainmatter
\chapter{绪论}

\section{研究背景和意义}
自动化地对分布式协议验证其正确性是一个重要且困难的挑战。
为了验证分布式协议的正确性,我们可以尝试证明分布式协议的每一个状态都满足一个预先给定的不变式,亦即安全属性(safety property)。
对于小型系统,我们可以通过遍历每个状态的方式进行验证。
然而,在工业实践中,分布式协议的参数众多且庞大,使得系统状态的数量巨大,我们无法简单地采用遍历的方式来验证安全属性是否在每个状态下都成立。
过去的研究中往往采用寻找一个蕴含着安全属性的不变式的方式来验证协议的正确性,这个不变式经过所有可能的状态转移后仍然能保持其自身的正确性。
这个不变式被称为归纳不变式。自动化地寻找或者生成分布式协议的归纳不变式是验证自动化分布式协议正确性的关键步骤。对于分布式系统,研究表明,给定一个正确的归纳不变式,几乎所有其他证明工作都可以自动完成。

\TLA \cite{TLA+}是一个对程序和系统,尤其对并发和分布式的程序和系统进行规约建模的高级语言。
在并发和分布式系统设计和开发过程中,非常容易发生基础性的设计问题,这些问题往往难以被发现。
而\TLA 以及其工具,利用集合论和时态逻辑精确地表达系统的状态和行为,可以帮助开发人员在设计阶段避免这些问题,以及在开发阶段定位问题。

目前,归纳不变式的自动生成技术,大多基于Ivy \cite{Ivy} 实现。
然而,Ivy的功能相比较\TLA 比较局限,且\TLA 在工业界的应用更加广泛。
当前针对\TLA 规约的自动化归纳不变式生成方法较少,且实现方法比较单一。
我们希望能\TLA 语言上开发出一种新的自动化归纳不变式生成方法,借助机器学习的技术以提高生成效率,为分布式协议的设计和验证提供帮助。

\section{研究问题}
本文使用 Python 和 Java 语言,实现对以\TLA 语言描述的分布式协议规约的归纳不变式的自动化生成工具 RLTLA,并通过实验对工具性能进行基准测试。
% TODO

\section{国内外研究现状}
目前的归纳不变式生成技术主要是基于 Ivy \cite{Ivy} 实现的,科研人员基于 Ivy 的平台设计了诸多归纳不变式自动生成的算法和工具,
基于 \TLA 的研究现在相对较少。

从实现理念和思路上,这些工具的大致可以分为两类,一种是基于程序语义(syntax-guided)的白盒技术,另一种是基于程序行为的黑盒技术。
近年来,随着 AI-for-SE的发展,一种叫做ICE\cite{ICE}(implication counterexamples)的学习框架流行起来,它将不变式的证明工作分为了两个部分:学习者和教育者。
依赖随机搜索、决策树\cite{garg2016learning}、强化学习\cite{LIPuS}等技术,许多工作推进了学习者模块的发展。
此外,也有人将新颖的语言大模型引入了不变式生成的工作中\cite{llm}。
白盒技术和黑盒技术的界限并不明确,一些工具其实兼而有之地采取两种技术的优势。

DistAI\cite{DistAI}以及DuoAI\cite{DuoAI}来自同一个研究团队,使用枚举候选不变式的算法进行自动不变式生成。
他们基于已有的小体积的运行数据,在削减过的空间上,在有限的句法空间中通过工具裁剪谓词来生成候选不变式,
也就是说对已有的运行数据,依赖协议中已有的谓词进行抽象,总结出候选不变式。然后对候选不变式进行验证并对失败的候选不变式继续裁剪,
最后按照一定顺序进行枚举,组合成为最终的归纳不变式;
I4\cite{I4}基于有限实例推广进行自动不变式生成。
I4会首先创建协议的多个有限实例,利用模型检查工具自动导出该有限实例的归纳不变式。
然后,I4尝试进行泛化,将不变式推广到更大的实例上,最终使之在协议上成立;
LIPuS 则在基于语义的基础上,使用了强化学习的框架对搜索空间进行剪枝,并在修剪过后的空间上进行 SMT 求解。使用这种方式可以有效地减少对SMT solver 的调用,从而提高生成归纳不变式的效率。

以上的工作,均是基于 Ivy 的。目前基于 \TLA 的归纳不变式生成工具较少。

endive\cite{endive}基于I4的思想,是第一个基于 \TLA 的归纳不变式生成工具。
endive 需要用户提供原子公式,算法就会根据这些原子公式自动生成可能的归纳不变式,并且从中排除掉那些违反安全属性的不变式。
之后,endive 会选择可以杀死反例最多的不变式,并重复这一过程,直到组合出最终的归纳不变式。

\section{本文组织结构}
本文的组织结构如下:


\chapter{预备知识}\label{chap:pre-knowleage}

本章节将以规约 \textit{Client\_Server} (图\ref{fig:client_server})为例,
介绍 \TLA 规约的基本结构,以及在寻找归纳不变式过程中的其他预备知识。
\begin{figure}
    \centering
    \includegraphics[width=0.8\textwidth]{figures/Client_Server.pdf}
    \caption{Client\_Server 规约}
    \label{fig:client_server}
\end{figure}

\section{\texorpdfstring{TLA\textsuperscript{+}}{TLA+}}
\href{https://lamport.azurewebsites.net/tla/tla.html}{\TLA} \cite{TLA}是由计算机科学家 Leslie Lamport 主导开发的,
用于对计算机程序和系统建模,尤其是对并行系统和分布式系统建模的高级语言。
它是基于使用简单的数学语言来精确描述系统行为的理念开发的。
因此,\TLA 的表达方式和一般的编程语言有很大的不同,反而和数学语言更为接近。
\TLA 并不是一种编程语言,而是一种规约语言,它不关注协议或者系统的具体实现,从而能更高层次看到程序整体的设计。
因此,\TLA 及其工具对于消除代码中很难发现和纠错成本高昂的错误非常有用。

需要注意到的是,\TLA 并不是为了寻找归纳不变式而设计的,而是为了对系统进行建模,是为了让规约开发人员能更好地表达一个协议。
它的语法更加丰富,以更加直观的方式表达一个协议。

开发者使用 \TLA 或者其他工具来对分布式协议进行建模的代码,我们将其称之为规约(specification,简称spec)。
图 \ref{fig:client_server} 展示了一个简单的 \TLA 规约,其中包含了一个简单的客户端和服务器的通信协议。
其中两个重要的谓词是 $Init$ 和 $Next$。
$Init$ 表示系统的初始状态,描述系统最开始时的状态;
而$Next$ 则是表示系统的状态是如何转移,也就是系统的状态在每个时间片后会发生怎样的变化。
在这个规约中,其他的谓词还有$Connect, Disconnect$ 等,定义这些谓词,就像在一般的编程语言中定义函数一样,方便阅读和重复使用。
一些规约中还有谓词 $TypeOK$,用于约束变量的类型。
另一种在自动归纳不变式生成研究中常常使用的工具,IVy,也有相似的语法和结构。
可以看到的是,\TLA 更关注系统的状态和系统状态是发生怎样的转移,对于系统状态转移的具体实现,\TLA 并不关心。
这样的描述方式和状态机非常相似。

谓词$Safe$ 是安全属性(safety property),一个正确定义的分布式协议规约,应当在每个可达的状态下都满足安全属性。
这个变量在自动化生成归纳不变式的研究中非常关键。

TLC 是 \TLA 集成的模型检测工具。
除了 TLC 以外,\TLA toolbox\cite{tla+toolbox} 还集成有PlusCal\cite{PlusCal}和TLAPS用于命题证明工具,sany用于语法检查工具,
tex 用于将\TLA 美化打印的工具等,这些工具与本文所讨论的问题相关性不高,不展开讨论。

本文所述工具接受\TLA 的规约。

\section{归纳不变式和归纳反例}
验证分布式协议的正确性,就是验证协议定义的安全属性(safety property)是否在每个可达的状态下都成立。
在 \textit{Client\_Server} 规约中,我们可以看到 $Safe$ 是一个安全属性,
它表达的是,在任何状态下,如果两个客户端同时连接有同一个服务器,那么这两个客户端是同一个客户端。
换言之,两个不同的客户端不能连接到同一个服务器。

对于简单的系统,即变量和状态不多的系统,我们可以通过遍历每一个可能的状态来验证。
但是对于稍微复杂一些的系统,尤其是越来越多的分布式系统,规模越来越大,状态也越来越复杂。
通过简单的遍历的方式来验证系统的正确性,是不现实的。
寻找一个能够蕴含安全属性的不变式,并且能够在所有可能的状态转移后保持其自身的正确性,这个不变式被称为归纳不变式。
以数学的语言表示为:
\begin{align}
    &Init \Rightarrow Ind \label{con:init}\\
    &Ind \land Next \Rightarrow Ind' \label{con:inductive}\\
    &Ind \Rightarrow Safe \label{con:safety}
\end{align}
其中$Init$ 表示初始状态,$Next$ 表示状态转移,$Safe$ 是安全属性,$Ind$ 表达的是归纳不变式,
而$Ind'$表达谓词$Ind$经过状态转移后的变量的状态。
定理\ref{con:init}表明归纳不变式在初始状态下成立;
定理\ref{con:inductive}表明归纳不变式在状态转移后依然成立,具有归纳性质。
比如说,如果$Ind$在状态$s$下成立,那么在$s$的后继状态下,$Ind$依然成立;
定理\ref{con:safety}表明归纳不变式蕴含安全属性,因此,如果某个运行时可达状态满足$Ind$, 那么也必然满足$Safe$。
这是我们寻找一个这样的归纳不变式的目的,通过归纳不变式的正确性验证安全属性的正确性。
这是归纳不变式所必须满足的三个条件。

\begin{align}
    &\left.A_{1} \triangleq \forall s \in  { Server }: \forall c \in  { Client }:  { locked }[s] \Rightarrow(s \notin { held }[c])\right) \\
    &{ Ind } \triangleq  { Safe } \wedge A_{1} \label{con:candidate_ind}
\end{align}

对于\textit{Client\_Server} 规约,表达式\ref{con:candidate_ind}是一个可能的归纳不变式。
可以看到的是,$Ind$是由$Safe$和$A_{1}$两个谓词逻辑表达式合取组成而来。
事实上,大部分规约的归纳不变式都可以表达为$Ind \triangleq Safe \wedge A_1 \wedge A_2 \wedge... \wedge A_n$的形式。
其中析取子式$A_k$是约束状态的谓词,我们将之称为引理不变式(Lemma Invariant)。
因为归纳不变式$Ind$是由这些引理不变式$A_k$组合而成的,也就是说,归纳不变式强于每一个引理不变式。
因此,引理不变式需要满足不变性,也就是在系统运行的每个状态下都成立,才能成为一个合适的引理不变式。
但是,引理不变式本身不需要满足归纳性,只需要他们和安全属性的析取结果能够满足归纳性。

对于一个谓词表达式$P$,如果一个状态$s$满足$s \models P$,但是$s$的后继状态$s_{n} \models \neg P$,
那便可以称$s$ 为 $P$的归纳反例(counterexample),揭示了$P$不是归纳不变式。
一个归纳反例往往包括两个状态,前一个状态满足谓词$P$,而后一个状态不满足谓词$P$。

\begin{figure}
    \centering
    \includegraphics[width=0.7\textwidth]{figures/ind-cti.pdf}
    \caption{归纳不变式和归纳反例}
    \label{fig:ind-cti}
\end{figure}
图\ref{fig:ind-cti}形象地介绍了归纳不变式和归纳反例。
寻找归纳不变式的过程,也可以理解为通过添加新的引理不变式,来排除归纳反例的过程。
但是,尤其是对于越复杂的系统而言,寻找归纳不变式并不是一个简单的任务。
实现归纳不变式的自动生成是形式化验证领域一个重要的研究目标,这也是本文研究的内容。

\section{TLC 和 Apalache}
TLC和Apalache是两个常见的面向\TLA 规约的模型检查工具(model checker)。
本文通过TLC和Apalache来验证生成的候选不变式的性质。
主要是借助TLC或Apalache获取归纳反例,以及验证生成模块生成的谓词表达式是否是有限实例上的不变式。

\subsection{TLC}
TLC既是对\TLA 规约的模型检查工具,也是一个面向规约的模拟器。
它是一个显式状态模型检查器,依照用户给出的规约和设置,搜索所有满足约束的状态和状态转移,
并在这个过程中检查安全属性和其他用户定义的谓词逻辑时时是否成立。
如果遇到错误,TLC会将错误的状态和状态转移过程输出,以便用户进行分析。

TLC可以通过使用超过32个计算机线程以获得近乎线性的加速。
它可以通过在分布式部署的计算机网络上运行来进一步加速模型检查,并提供在云系统上的轻松部署。

\subsection{Apalache}
Apalache\cite{apalache1, apalache2} 和 TLC 不同的是,Apalache 并不是通过遍历所有可能的状态来检验安全属性是否成立,而是通过 SMT solver 来检验。
它是将\TLA 规约转换为 SMT 问题,然后使用 SMT solver (如Z3\cite{z3})求解来检验安全属性是否成立。
Apalaches 是一种符号检查器,它和 SMT solver 一样基于逻辑推理和公式求解实现的。
Apalache 对\TLA 源文件的语法中引入了一些限制,
尽管没有完全支持\TLA 的所有语法,但是这方便使用 SMT 求解器进行求解。

\TLA 是一个“弱类型”的编程语言,它对变量没有严格的类型注明。
但是,Apalache 需要了解\TLA 规约中变量的类型才能工作。
尽管 Apalache 有一套自己的类型推断系统,但是,它并不能完全解决所有的类型推断问题。
这使得用户,对于某些协议,需要以注释的形式来提供变量的类型,才能交给Apalache进行处理。

% 补充关于 返回结果和CTI的内容?

\section{强化学习}
强化学习(Reinforcement learning, RL)\cite{rl}是机器学习的一个领域,强调如何基于外部环境做出决策,以获得最大化的预期累积奖励。
是区别于监督学习和非监督学习的另外一种基本的机器学习方法。
强化学习的关注点在于寻找对未知领域的探索和对已有知识的利用之间的平衡。
它的目标是通过奖惩来控制智能体完成任务,以获得最大化的预期累积奖励,但程序无需明确告诉智能体如何完成任务。

在机器学习问题中,环境通常被抽象为马尔可夫决策过程(Markov decision processes,MDP)\cite{markov},
因为很多强化学习算法在这种假设下才能使用动态规划的方法。
传统的动态规划方法和强化学习算法的主要区别是,后者不需要关于MDP的知识,而且针对无法找到确切方法的大规模MDP。

\begin{figure}[h]
    \centering
    \includegraphics[width=0.6\textwidth]{figures/Reinforcement_learning_diagram.pdf}
    \caption{强化学习框架}
    \label{fig:rl}
\end{figure}
图 \ref{fig:rl} 展示了强化学习的框架。
强化学习中,主要是智能体(agent)和环境(environment)之间的交互。
环境是智能体所处的环境,它会根据智能体的动作,给予智能体奖励或者惩罚,并作出状态转移。
智能体根据奖励和环境状态的变化来调整自己的策略。
智能体,可以感知环境的状态(State),并根据反馈的奖励(Reward)学习选择一个合适的动作(Action),来最大化长期总收益。
智能体会根据环境的反馈,调整自己的策略,以获得最大化的预期累积奖励。
实现强化学习的策略算法有很多,其中最著名的有 Deep Q Network(DQN)\cite{dqn},Q-learning\cite{q-learning}等。

在本文的项目中,我们使用强化学习的方法来加速归纳不变式的生成。
我们使智能体理解\TLA 原文件的内容,让智能体合理选择生成归纳不变式的种子(seed),
并将每一次智能体选择的种子所生成的不变式的检验结果反馈给智能体,包括反例的数量,内容和生成时间等。
智能体根据这些反馈信息,调整自己的策略,以便更快地找到一个满足安全属性的归纳不变式。



\chapter{归纳不变式自动生成工具的设计}

总体上,我们的归纳不变式生成工具包括一下几个步骤:
\begin{itemize}
    \item 解析用户输入,生成用于检测三个属性的配置文件(.cfg)。
    \item 检验模块对生成模块所生成的候选不变式检验其正确性、独立性和与已有不变式析取结果的递归性,
    并解析模型检测器返回的结果,回传给生成模块;同时检验模块还需存储合适的不变式,并在得到归纳不变式时,输出结果。
    \item 基于用户输入和检验模块的结果,生成合适的不变式。
\end{itemize}

rlTLA的工作流如图\ref{fig:rltla}所示,主要分为候选不变式生成模块(Invariant Generator),候选不变式检验模块(TLC/Apalache)两个部分。
其中候选不变式生成模块接入了强化学习,训练强化模型智能体,以提高候选不变式的生成效率和准确率。
候选不变式检验模块则接入了TLC和Apalache,对生成的候选不变式进行验证,检验其正确性,独立性和与已有不变式析取结果的递归性,并将结果返回给生成模块。
在初始化阶段,候选不变式检验模块还需要生成用于检测这三种属性的配置文件(.cfg)。
目前系统接受 endive 提供的数据源,使用 endive 中的对规约人工标记的谓词,作为候选不变式的种子(seed)。
系统的输出是对一系列候选不变式的析取范式,对于系统而言,是一个包含 $Safety$ 属性的归纳不变式。

基本的逻辑结构如伪代码\ref{alg:rltla-workflow}展示。
初始化时候,候选的归纳不变式首先析取$Safety$属性,然后生成候选不变式的CTI(Counterexample to Induction)。
在每一轮的迭代中,强化学习系统都会不断地生成一个个候选不变子式,直到生成的子式在系统运行环境中是不变式,且不会被已有的不变式所包含。
这样的不变式便会加入到候选的归纳不变式中做析取,直到对候选归纳不变式生成的CTI为空,即不再有反例产生。
这说明现在系统中所有不变式的析取结果是一个包含$Safety$属性归纳不变式,系统选择将这个结果输出给用户。


\begin{figure}
    \centering
    \includegraphics[width=0.7\textwidth]{figures/workflow.pdf}
    \caption{rlTLA-workflow}
    \label{fig:rltla}
\end{figure}

\begin{algorithm}
    \caption[short]{workflow of rlTLA}
    \label{alg:rltla-workflow}
    
    \begin{algorithmic}[1]
        \REQUIRE \ \\
        $M$: Finite instance of parameterized system\\
        $Safe$: Safety property
		\ENSURE \ \\
        $Ind$: Inductive invariant
		\STATE $Ind \gets Safe$
        \STATE $IndCTIs \gets GenrateIndCTIs(M, Ind)$
        \WHILE{$IndCTIs$ is not empty}
            \STATE $InvCTIs \gets IndCTIs$
            \WHILE{True}
                \STATE $Inv \gets GenerateCandidateInvariant(M, IndCTIs)$
                \STATE $InvCTIs \gets GenrateInvCTIs(M, Inv)$
                \IF{$InvCTIs$ is not empty}
                    \STATE $Continue$
                \ENDIF
                \IF {not $CheckDerivation(Ind, Inv)$} 
                    \STATE $Ind = Ind \wedge Inv$
                    \STATE $Break$
                \ENDIF
            \ENDWHILE
            \STATE $IndCTIs \gets GenrateIndCTIs(M, Ind)$
        \ENDWHILE
        \RETURN $Ind$
    \end{algorithmic}
\end{algorithm}

\section{候选不变式生成模块}

对于一个给定的规约的有限大小的实例,候选不变式生成模块的工具就是生成一系列在规约的每个状态下都成立的候选不变式,这些不变式以一阶逻辑谓词的形式存在。
为了找到这些不变式,我们使用一种基于语法合成指导的归纳不变式生成技术。
我们从输入的种子(seed)谓词中,按照强化学习模块的意愿选择一部分种子谓词,并按照语法组合成一个可能的不变式。
每个种子谓词都是针对系统状态变量的原子布尔谓词。
当然,我们也有可能采取给出种子谓词的否定,根据\TLA 的语法,我们只需要在给出的谓词前面加上否定符号 \textbf{“\~{}”}即可。
这部分的决定权交给强化学习模块,强化学习模块会根据当前的状态,选择一个合适的种子谓词,或者它的否定。

一个不变式的语法大致可以表达为:
\begin{align}
    <lemma> &: = <quant>:<expr>   \\
    <quant> &:= \forall x\ \backslash in\ SetA | \exists y\ \backslash in\ SetB \\
    <expr>  &:= <seed>| \sim<seed> | <seed> \wedge <seed>
\end{align}
注意到本文中所提及的大部分的\TLA 规约都带有存在量词和全程量词,我们直接将这一部分量词当作输入的一部分。
尽管可以带来多余的量词表达式,但是这部分量词并不会对整个谓词的布尔值带来实际的影响,这部分多余的量词可以不做处理。
另外,我们对于候选不变式内部每个小的谓词之间的连接方式统一选择了合取符号“$\vee$”。
这是因为,一方面在逻辑表达上已经足够完备\cite{or-complete},另一方面,这也可以帮助了我们简化问题,缩小了搜索不变式的空间。


\section{强化学习在生成模块中的应用}

本文的目的是希望研究强化学习在自动的归纳不变式生成过程中的应用。
强化学习是一种通过智能体和环境的交互,智能体通过观察环境的状态,采取行动,获得奖励,来学习如何在环境中获取最大的奖励。
在本文中,强化学习被应用于生成归纳不变式的各个析取子式,也就是候选不变式。

强化学习模块接受\TLA 协议和检验模块对于本身生成的候选不变式的检验结果,修改策略,生成更加合理的候选不变式。
强化学习模块的输入是一个状态,输出是一个动作,动作是对于种子谓词的选择与否,和是否选择它的否定。

强化学习模块的目标是生成一个合适的候选不变式,这个候选不变式在规约的每个状态下都成立,且不会被已有的不变式的析取结果包含。
这个过程是自动化的,但人可以通过挑战对强化学习智能体在每个状态下的每个动作的选择给出合适的奖励或者惩罚,以指导智能体学习到
一个合适的策略,并应用到后续的候选不变式生成中。

最高的奖励应当给予给出最终的归纳不变式的行为,也就是给出了一个不变式,使得它和前面所有不变式的析取结果是归纳不变式的行为。
当强化学习模块给出一个不变式的时候,也应当给出一定的奖励。给出一个不变式,尤其是给出第一个不变式,往往是一个十分困难的过程。
而且,这一步也是实现最终目标的关键。因此,我们应当给予一定的奖励,以鼓励智能体继续学习。

生成一个已经被已有不变式包含的不变式,尽管没有意义,但也体现出智能体如何寻找不变式的能力,因此也应当给予一定的奖励。
这个奖励的值很小,但是也是必要的。因为这个过程是一个逐步的过程,智能体需要不断地尝试,才能找到一个合适的不变式。
但是,为了防止智能体的惰性,我们不允许智能体在已有的不变式上简单的合取上一个谓词,然后给出这个谓词作为候选不变式。
这是简单的重复,是绝对没有意义的,我们对智能体的这种行为应当给予惩罚。


\section{候选不变式检验模块}

候选不变式检验模块需要调用模型检查器,并且需要将输出解析,并将结果返回给生成模块。
对于候选不变式的检验,可以分为三个部分,分别是正确性、独立性和多个不变式析取结果的递归性。
引理\ref{con:inv_correct}表达了候选不变式的正确性,即候选不变式在规约的每个状态下都成立。
引理\ref{con:inv_indepence}表达了候选不变式的独立性,即新生成的候选不变式不能被已有的不变式的析取结果包含。
\begin{align}
    &Spec \triangleq Init \wedge [Next] \\
    &Spec \vDash Inv \label{con:inv_correct} \\
    &Spec \wedge IndCand \nvDash Inv \label{con:inv_indepence}
\end{align}

检验候选不变式的正确性,就是检验每个候选不变式是否在规约的每个状态下,布尔值都为真。
这个问题虽然简单,但是直觉上可能需要遍历许多状态和多个状态转移轨迹,可能需要很长的时间。
但是实际上,很多的不正确的候选不变式在规约的某个比较容易到达的状态被验证器检验出来,便可以退出了。
一个正确的不变式,尽管现在还不能成为归纳不变式,但是它确实我们需要归纳不变式的开始。

检验不变式的独立性时,我们需要验证新生成的候选不变式是否能被已有的不变式的析取结果包含。
检验不变式的独立性十分重要,尤其是对于指导强化学习生成候选不变式,
因为如果新生成的候选不变式能被已有的不变式的析取结果包含,那么生成模块为了得到更高的奖励,
会多生成这样的候选不变式,这样会导致不变式的重复,析取的结果的约束能力也不能加强,对状态空间不能做出有价值的修剪,
这样的不变式便是没有意义的。系统也就无法找到一个合适的归纳不变式。

检验归纳不变式的递归性,是我们工作的终点。
如果多个不变式的析取结果具有递归性,那么我们可以将这个析取结果作为归纳不变式,我们可以将这个结果输出,并结束这个循环。
对于不变式的递归性质,在引理\ref{con:init}和\ref{con:inductive}中已经提及。
归纳不变式需要包含所有的初始状态,并且,从归纳不变式约束的状态出发,进行状态转移,新生成的状态也需要满足归纳不变式。
另外,我们最根本的目的是证明安全属性,那么,归纳不变式需要蕴含安全属性。



\chapter{归纳不变式自动生成工具的实现}\label{chap:implementation}

本章节将基于设计方案,详细介绍了归纳不变式自动生成工具的实现细节,包括模块之间的交互,以及模块的具体实现。

\section{候选不变式检验模块}

候选不变式检验模块主要的职责是对生成模块的生成的候选不变式的正确性,给出的候选归纳不变式的递归性,以及对新生成不变式的独立性进行判断。

候选不变式检验模块接入了TLC和Apalache,用户可以选择其一对生成的候选不变式进行验证。
TLC 和 Apalache 是两个常见的面向\TLA 规约的模型检查工具(model checker),可以使用相似的配置文件对规约进行验证,
但是,两者的结果输出格式不同,需要做分别处理。

由于Apalache需要用户对协议中的变量和常量做出类型的注释,因此,目前能够提供的测试集中大多数的规约都无法使用Apalache进行验证。
在系统实现时,我们默认状态下使用TLC作为系统的模型检查器。
当然,在条件允许时,用户可以设置系统的参数来使用Apalache作为系统的模型检查器。

在检验候选的不变式和归纳不变式时,如果出现反例,无论是对候选不变式的反例,还是对归纳不变式的归纳反例,都应该提取出来,返回给生成模块。

\subsection{model checker的配置文件和运行选项}

对于图\ref{fig:client_server}中的规约,TLC 和 Apalache 会使用默认的配置文件进行验证,
即以 $INIT$为初始状态,$NEXT$ 为状态转移关系,在状态变化的过程中验证 $Safe$ 安全属性的正确性。
用户也可以指定使用其他配置文件,以验证从不同状态出发和不同状态转移条件下的用户定义的不变式的成立与否。
比如说需要验证的不变式,可以放在$INVARIANT$ 字段下。
对于一些常量,用户也可以通过配置文件$CONSTANTS$字段进行定义。
一个典型的配置文件如下:
\begin{lstlisting}
INIT Init
NEXT Next

INVARIANTS Inv_0 Inv_1 Inv_2

CONSTANTS
...
\end{lstlisting}
我们可以简单的理解为,模型检测器可以判断$INIT \wedge NEXT \vDash INVARIANT$ 是否成立。
在不成立时,模型检测器会给出一个状态的链接,展示系统状态如何从初始状态转移到不满足不变式的状态。

我们希望TLC和Apalache为我们验证生成模块生成的候选不变式的正确性,独立性和与已有不变式析取结果的递归性。
在验证过程中,我们希望模型检查器能够输出验证结果,以及验证过程中的反例(Counterexample)。

验证不变式的正确性是验证这三种性质中最为简单的。
只需要将需要验证的候选不变式放入$INVARIANT$字段中,然后运行模型检查器即可。
如果没有报错,说明候选不变式在规约的有限实例上保持布尔值为真。

由于候选的归纳不变式是由一系列引理不变式和安全属性析取而来,且每一个析取子式都满足$Init \wedge Next \vDash Lemma$,
所以候选归纳不变式自然满足引理\ref{con:init}和\ref{con:safety}。
验证候选的归纳不变式的递归性质时,我们只需要验证引理\ref{con:inductive}的正确性,我们需要验证归纳不变式在状态转移后依然成立。
在此过程中,模型检查器弹出的报错就是归纳反例。

验证不变式的独立性是验证这三种性质中最为困难的。
检查不变式的独立性就是检查新生成的引理不变式是否能杀死(eliminate)候选的归纳不变式的归纳反例。
借助TLC,我们将系统的初始状态设置为这些归纳反例的状态之一,也就是将$INIT$ 设置为这些归纳反例状态的析取。
然后看在这些状态下,哪些新生成的引理不变式不成立。不成立便能代表他们能够杀死对应的归纳反例。
在操作中,我们需要用变量记录下这些引理不变式的值。
使用TLC的\textbf{"-dump"}选项将每个状态下的变量值输出到文件中,然后解析这些文件,找到能够杀死归纳反例的引理不变式。
如果归纳反例太多,会导致TLC计算状态转移关系的时间过长,状态转移图过于复杂。
尽管可以通过\textbf{"-workers"}选项添加TLC调用的线程数量,但是也收效甚微。
因此将归纳反例分组,分别使用一个线程对每一组归纳反例进行验证,可以有效减少验证时间。

但是这种做法并不充分,还需要通过$IndCand \wedge Next \nvDash Lemma$的结果来确认新引理不变式的独立性。


我们需要将新生成的不变式放到一个新的文件中,并使用关键字\textbf{EXTENDS} 将原有规约中的定义引入。
这样,我们就可以在新的文件中使用原有规约中的定义,和引入生成模块生成的候选不变式。
由于新的\TLA 文件中有着相似的结构,在验证同一个性质时,配置文件是可以复用的。
所以我们在系统运行之初就定义好配置文件中的内容,并写入硬盘供TLC/Apalache使用。

在使用TLC验证时,我们还需要关注诸多选项。
\textbf{"-config"}是我多样化使用TLC和apalache的关键,通过这个选项,我们可以指定TLC和Apalache的配置文件,以检验不变式的不同性质。
\textbf{"-deadlock"}选项用于检查是否存在死锁状态,如果选择了这个选项,那么TLC就不会检验死锁。
由于我们的目的是检验不变式的一些性质,所以我们不需要检验死锁,并选择了这一选项。
\textbf{"-continue"}选项揭示了TLC在检测出错误后是否继续运行,为了得到更多的反例,我们需要使用这个选项。

\subsection{model checker 的调用和结果解析}
本项目的代码主要基于Python实现,然而不论是TLC还是Apalache,都是Java实现的模型检查器,且没有可以直接调用的Python接口。
因此,我们需要通过Python的subprocess库来调用Java程序,并通过解析Java程序的命令行输出结果来获取验证结果。
在验证不同性质的时候指定好不同的配置文件并调整好不用的运行参数。

对于结果的解析,主要是将TLC或Apalache的输出结果进行解析,去除无用的信息,将有用的信息交给生成模块,以便强化学习模块调整策略,提高生成的候选不变式的正确性。
TLC和Apalache尽管两者有着不同的输出格式,但是他们的功能其实是一致的,都是将出现不变式错误时的状态,以及前序状态,也就是错误轨迹(error trace)。
错误轨迹的每一个节点都是一个状态,表达的是在这个状态下,各个变量的值。
TLC会以析取范式的形式将各个变量的值表达出来,而Apalache默认使用的json文件格式,将各个变量的值以键值对的形式表达出来。
我们需要将这些信息解析出来,以便强化学习模块能够理解这些信息,调整生成的候选不变式。

不变式反例和归纳反例有着相似的作用,都是用于提示用户或者生成模块,哪些状态下,不变式或者归纳不变式不成立。

类CTI的信息如图\ref{fig:class_cti}。在模型检测器检测去不变式或者归纳不变式的错误时,
通过类CTI中的静态方法\textbf{parse\_cti},可以将错误轨迹提取出来,生成多个不同的cti对象。
\begin{figure}[h]
    \centering
    \includegraphics[width=0.3\textwidth]{figures/class_cti.pdf}
    \caption{CTI 类信息}
    \label{fig:class_cti}
\end{figure}
CTI类反应的是一个状态,在这个状态下,所有变量的值存储在\textit{cti\_str}字段中。
从这个状态出发,系统可以运行一个状态,这个状态下,给出的候选归纳不变式不成立。

CE类有着和CTI类相似的结构,不同的是,CE类更关注那个个使得不变式不成立的状态。


\section{候选不变式生成模块}

生成模块是本项目的关键,它负责生成候选不变式,检验模块是为生成模块服务的。
不同于以往的归纳不变式生成工具,使用随机枚举的方式生成候选不变式,我们引入强化学习来提高我们枚举的效率和成功率。
强化学习是一种通过智能体和环境的交互,智能体通过观察环境的状态,采取行动,获得奖励,来学习如何在环境中获取最大的奖励。
如图\ref{fig:rl}所示,一个强化学习模块可以分为智能体和环境两个部分。

\subsection{强化学习环境实现}

环境接收智能体的行动,做出反应并返回自身的状态和奖励,智能体根据环境的状态和奖励,调整滋生的策略,做出行动的选择。
本文使用\href{https://gymnasium.farama.org/}{gymnasium} \cite{gymnasium} 实现强化学习的环境。
gymnasium是一个开源的强化学习环境,提供了标准的环境接口,方便用户实现自己的强化学习环境,为我们实现环境提供了诸多便利。

自定义一个基于 gymnasium 的环境,最重要的是定义好环境的 \textbf{action\_space} 和 \textbf{observation\_space},以及实现step 函数。

\textbf{action\_space} 和 \textbf{observation\_space} 的数据类型都基于gymnasium的Space类,这其中包括众多类型。
\textbf{action\_space} 代表智能体所能采取的行动的空间,我们这里选择了\textbf{MultiDiscrete},智能体选择的行动为一个等同如输入的 \textbf{predicates} 长的向量。
这个向量每个维度的值都只能是0, 1, 2。在后续,我们会给这个向量 -1,得到一个每个维度上只有-1, 0, 1的向量,这个向量代表了我们的候选不变式。
每个维度上的值代表着对应的谓词是否被选择,-1代表否定,0代表不确定,1代表肯定。
向量与谓词相乘,便可以得到一个谓词表达式,再在前面加上定义好的量词,便可以得到一个候选不变式。

\textbf{observation\_space} 代表智能体所能观察到的环境的空间。
这里选择了\textbf{Dict} 类型。给予智能体参考的信息有:当前的不变式,对于当前不变式的反例或者归纳反例,可以选择的谓词,已经加入归纳不变式的不变式。
\textbf{Dict} 中每个键所对应的值也必须是一个Space类型,对于这四条信息,都选择了\textbf{Text}类型,也就是对应于字符串。

\textbf{step} 函数是环境的核心,它接受智能体的行动,返回环境的观察值,奖励和是否终止。
在我们的环境中,智能体的行动是一个向量,我们需要将这个向量转化为一个候选不变式,然后交给检验模块进行验证。
如果得到了归纳不变式,会给予100的奖励并退出。除此以外的所有情况,因为没有得到归纳不变式,所以都不会推出。
如果得到一个合适的不变式,会给予10的奖励,如果得到了一个错误的不变式,会给予-10的奖励。
如果得到一个不变式,但是不“独立”,会给予2的奖励,但是如果智能体简单地将正确的不变式合取上一个新的谓词,会给予-2的奖励。
与此同时,\textbf{step} 函数还需要返回观察值,这对应于observation\_space中的四条信息的数据结构。
这里复用了反例字段,在得到合适不变式时,返回归纳反例,否则,返回的是智能体提供的不变式的反例。
% 配个图
\begin{table}[!h]
    \label{table:award_punish}
	\centering
	\caption{奖励与惩罚的情况设置}
	\label{tab::situation}
	\renewcommand\arraystretch{1.4}
	\begin{tabular}{p{0.25\textwidth}p{0.15\textwidth}p{0.5\textwidth}}
		\toprule
		\textbf{候选不变式结果}   & \textbf{奖惩值} \\ 
        \midrule
		简单重复已有不变式 & -10 \\
		非不变式      & 0   \\
		被已有不变式覆盖  & 2   \\
		归纳不变式子式   & 10 \\
        \bottomrule
	\end{tabular}
\end{table}

\subsection{强化学习智能体实现}

实现智能体并不困难,gymnasium 的环境接口可以对接诸多的强化学习智能体实现框架。
本文选择了OpenAI \href{https://github.com/openai/baselines}{baselines}\cite{baselines}来实现智能体部分。
baselines 包括多种强化学习算法。
本文尝试了多种算法,包括DQN,PPO,A2C等算法。

\section{非功能模块}

非功能模块包括日志功能,计时功能,报错信息等,这些功能伴随着系统每个行为,为开发人员和用户提供更多信息,方便调试和使用。

\chapter{运行试验和结果分析}\label{chap:run-analysis}

\section{部分运行结果展示}
\textbf{simple\_election}
\begin{lstlisting}
    Ind == 
    /\ TypeOK
    /\ Safety
    /\ \A VARS \in Acceptor : \A VART \in Acceptor : \A VARPA \in Proposer : \A VARPB \in Proposer : \E VARQ \in Quorum :   \/ (VARPA \in start)  \/ (VARPB \in start)  \/ (VARPA \in leader)  \/ (~(ChosenAt(VARQ,VARPB)))
    /\ \A VARS \in Acceptor : \A VART \in Acceptor : \A VARPA \in Proposer : \A VARPB \in Proposer : \E VARQ \in Quorum :   \/ (VARPB \in start)  \/ (~(<<VARS,VARPB>> \in promise))  \/ (DidNotPromise(VART))  \/ (ChosenAt(VARQ,VARPA))
    /\ \A VARS \in Acceptor : \A VART \in Acceptor : \A VARPA \in Proposer : \A VARPB \in Proposer : \E VARQ \in Quorum :   \/ (<<VARS,VARPA>> \in promise)  \/ (~(<<VARS,VARPB>> \in promise))  \/ (~(VARPA \in leader))  \/ (ChosenAt(VARQ,VARPA))
    /\ \A VARS \in Acceptor : \A VART \in Acceptor : \A VARPA \in Proposer : \A VARPB \in Proposer : \E VARQ \in Quorum :   \/ (VARPA \in start)  \/ (~(VARPA \in leader))  \/ (~(VARPB \in leader))  \/ (~(ChosenAt(VARQ,VARPA)))
    /\ \A VARS \in Acceptor : \A VART \in Acceptor : \A VARPA \in Proposer : \A VARPB \in Proposer : \E VARQ \in Quorum :   \/ (VARPA \in start)  \/ (<<VART,VARPB>> \in promise)  \/ (~(VARPA \in leader))  \/ (VARPB \in leader)
    /\ \A VARS \in Acceptor : \A VART \in Acceptor : \A VARPA \in Proposer : \A VARPB \in Proposer : \E VARQ \in Quorum :   \/ (~(<<VART,VARPB>> \in promise))  \/ ((VARPA=VARPB) /\ promise = promise)  \/ (~(VARPA \in leader))  \/ (~(ChosenAt(VARQ,VARPA)))
    /\ \A VARS \in Acceptor : \A VART \in Acceptor : \A VARPA \in Proposer : \A VARPB \in Proposer : \E VARQ \in Quorum :   \/ (~(<<VARS,VARPB>> \in promise))  \/ (<<VART,VARPB>> \in promise)  \/ (~(VARPA \in leader))  \/ (ChosenAt(VARQ,VARPA))
    /\ \A VARS \in Acceptor : \A VART \in Acceptor : \A VARPA \in Proposer : \A VARPB \in Proposer : \E VARQ \in Quorum :   \/ (~(VARPB \in start))  \/ (~(VARS=VART /\ promise = promise))  \/ (~(VARPA \in leader))  \/ (ChosenAt(VARQ,VARPA))
    /\ \A VARS \in Acceptor : \A VART \in Acceptor : \A VARPA \in Proposer : \A VARPB \in Proposer : \E VARQ \in Quorum :   \/ (VARPB \in start)  \/ (~(<<VART,VARPA>> \in promise))  \/ ((VARPA=VARPB) /\ promise = promise)  \/ (~(VARPB \in leader))
    /\ \A VARS \in Acceptor : \A VART \in Acceptor : \A VARPA \in Proposer : \A VARPB \in Proposer : \E VARQ \in Quorum :   \/ (VARPA \in start)  \/ (~(<<VART,VARPA>> \in promise))  \/ (~(VARPB \in leader))  \/ (~(DidNotPromise(VARS)))
    /\ \A VARS \in Acceptor : \A VART \in Acceptor : \A VARPA \in Proposer : \A VARPB \in Proposer : \E VARQ \in Quorum :   \/ (VARPB \in start)  \/ (~(<<VART,VARPB>> \in promise))  \/ (VARS=VART /\ promise = promise)  \/ (~(VARPA \in leader))
    /\ \A VARS \in Acceptor : \A VART \in Acceptor : \A VARPA \in Proposer : \A VARPB \in Proposer : \E VARQ \in Quorum :   \/ (VARPB \in start)  \/ (VARPA \in leader)  \/ (~(VARPB \in leader))  \/ (~(DidNotPromise(VARS)))
    /\ \A VARS \in Acceptor : \A VART \in Acceptor : \A VARPA \in Proposer : \A VARPB \in Proposer : \E VARQ \in Quorum :   \/ (~(<<VART,VARPA>> \in promise))  \/ (~(<<VART,VARPB>> \in promise))  \/ (VARS=VART /\ promise = promise)  \/ ((VARPA=VARPB) /\ promise = promise)
\end{lstlisting}

\textbf{lockserver}
\begin{lstlisting}
    Ind == 
    /\ TypeOK
    /\ Inv
    /\ \A VARS \in Server : \A VARC \in Client :   \/ (~(semaphore[VARS]))  \/ (~(VARS \in clientlocks[VARC]))  \/ (clientlocks[VARC] = {})
\end{lstlisting}

\section{运行结果分析}
endive 对于规约\textbf{simple\_election}的验证结果见于附录\ref{app:endive_simple_election}。

对比endive的结果,我们的工具可能会生成了13个不变式,而endive 则只生成了3个不变式。
相较而言,我们的工具生成归纳不变式寻找的引理不变式更多,效率较低。
对于这一现象的现象的解释,我认为是强化学习智能体在一开始尝试时会更偏向于选择已有的不变式比较相似的不变式,
但是在系统的提示下,相似的不变式往往不能得到很好的奖励值,于是系统便会在更加稀疏的区域寻找不变式。
而endive则是通过基于候选不变式能杀死归纳反例的个数进行选择,事实上,它可能验证的不变式的个数更多。
这样的结果导致了,我们的工具会生成更多的相近不变式,但是,这不妨碍我们的工具生成最终正确的归纳不变式。

另外,系统运行的总体时间相较于endive偏长。
这是因为TLC和apalache没有直接的python接口,我们需要通过调用命令行的方式来调用TLC和apalache,并通过字符串解析命令行输出。
一次校验的时间,平均在0.3秒左右。相较于endive 一次校验数十个不变式,我们的工具一次校验一个不变式,所以时间上会有所增加。
但相较于随机遍历,以\textit{simple\_election}为例,拥有12个用户定义谓词,对于长度小于4的候选不变式,大概需要遍历接近万次。
然而我们的工具,遍历的次数在千次左右,在效率上有显著的提升。




\chapter{总结与分析}
本章对论文工作进行了总结,并展望了未来可能的优化和改进方向。
\section{工作总结}
本文主要目的是验证机器学习,尤其是强化学习在对分布式系统规约的归纳不变式生成领域中的可行性和有效性。

本文在\TLA 的语言平台上,实现了一个基于强化学习的归纳不变式生成系统,
介绍了系统设计的预备知识和理论基础,以RLTLA的系统体系结构的设计和实现。

\TLA 相较于 IVy 更加复杂,其中存在灵活多样的数据结构,同时也支持任意的嵌套来表达规约。
这对开发人员在设计阶段表达系统的行为和状态转移关系提供了很大的便利,但这点对于归纳不变式生成工具的设计并不友好。

实现上,本文依靠于\TLA 源文件和 endive 对于\TLA 解析和人工识别的假设作为输入,基于 try-and-error 的生成思路,
采用强化学习的方式生成候选不变式,通过模型检查器验证候选不变式的正确性,独立性以及当前所有候选不变式析取结果的递归性,最终生成归纳不变式。
这一生成思路和大部分的归纳不变式生成工具类似,但是在实现上,本文寄希望于强化学习以提高生成效率。

本文使用 gymnasium 实现强化学习的算法,并借助 tianshou 提供的强化学习算法,实现完整的强化学习模块。
强化学习模块接受TLC或Apalache的反馈,基于已经给出的假设和\TLA 源文件生成候选不变式,并交给模型检查器验证。
gymnasium 提供了标准的环境接口,是目前十分受欢迎的环境开源工作,方便了强化学习环境的实现,并且可以接入多种强化学习算法。
tianshou 是一个完全基于 Pytorch 的强化学习框架,方便我们使用多种算法对强化学习模块进行训练和比较。

本文使用TLC和Apalache对生成的候选不变式进行验证,检验生成的候选不变式的正确性,独立性和与已有不变式析取结果的递归性,并将结果返回给强化学习模块。
TLC和Apalache没有提供python的接口,本文通过调用命令行的方式调用TLC和Apalache,并通过对命令行结果的解析,获取验证结果。

在测试部分,本文使用 endive 提供的测试用例对系统进行测试,并和endive进行了比较,
验证了强化学习在面向分布式系统规约的归纳不变式生成工作中具有可行性和有效性。

\section{未来展望}
由于时间和能力的限制,本文所实现的系统的性能和实现方式上有许多不足,存在大量的改进空间。

目前,和endive一样,RLTLA还以来于一些人工的输入,即一些人工识别的假设(predicates)
,这些假设的得到是依赖于人脑对于 \TLA 规约理解,尤其是对每个 Action 进行调用时参数类型的理解。
如果要自动化地识别和生成这些假设,也是一个复杂的工作,目前系统还不具备这项能力。
与此同时,人脑的参与可能带来效率的下降和不可预计的错处的出现。
未来需要实现一个功能更加丰富的静态分析工具,以自动提取出 \TLA 规约的语义信息,包括 Action 的参数类型等,
以帮助强化学习模块更好地理解 \TLA 规约和生成合适的候选不变式,或者通过接入大模型的方式,完成归纳不变式的生成。
另一方面,人工的输入也约束了可搜索的空间的大小,在这一条件下,
如同 endive 的通过机械搜索的方式,可能获得比强化学习更快的生成效率。

其次,目前提供给系统可以选择的谓词的范围基于\TLA 已经定义的最高层次的谓词,并没有考虑谓词的子式以及谓词之间的关系。
系统也无法考虑给出的这些谓词表达式之间,以及每个可能的候选不变式之间的关系。
目前系统设计上,一次只生成一个可能的候选不变式,
这两方面因素,导致了对于每一次的生成的候选不变式的检查,都需要调用1-3次模型检查器进行检查,这往往很花时间,导致系统的效率较低。

基于\TLA 的归纳不变式的生成是一个复杂的问题,在这一领域的研究不是十分充足,可以参考和对比的工作较少。
目前,对于基于\TLA 的归纳不变式生成工具还没有统一的测试集合,也没有十分充足的测试用例。
这导致我们一方面很难评估系统的效率,另一方面,也很难提供给强化学习模块足够的训练数据。
在 endive 的测试集合下,一个归纳不变式常常只需要不超过10个子式析取而来。
在这一背景下,强化学习的效率并不理想,常常带来相较于 endive 等工作提供的搜索算法更高的开支和更低的效率。



% 生成参考文献页
\printbibliography

\begin{acknowledgement}
四年以后,当写完本科毕业设计,即将结束本科生活时,张继华同学一定会想起,2020年9月3日,第一次走进南京大学仙林校区的大门,穿过教学楼和宿舍之间的主干道,报到注册的清晨。

时光终究是匆匆的,本科四年虽短暂,却不是一段单调或平淡的旅程。四年里,我收获了知识、技能、觉悟、友谊以及面向未来的勇气。
    
知识和技能是立足社会的根本。在南京大学这个广阔而高大的平台上,在优秀老师和同学的帮助下,我不仅学习了计算机科学的基础知识和技能,更重要的是,我学会了如何学习和思考。特别是魏恒峰老师,他悉心指导和耐心教诲让我在毕业设计中受益匪浅。他引领我进入分布式系统的世界,让我对计算机科学有了更深刻的理解。同时,邓楚宸同学在项目实现上做出了巨大贡献,在毕设完成过程中付出了巨大努力。
    
我出生在一个农民家庭。自2006年国家免除农业税起,到2008年免除义务教育阶段学杂费并给予农村贫困学生补助以来,这些政策极大地减轻了家庭负担,使我得以接受更好更长久的教育。高考的公平也为我提供了重新获得高等教育机会。党和国家塑造了今天的我。没有共产党,我可能会像祖辈一样成为一名农民。是党和国家给了我改变命运的机会。因此,我深感党的伟大,并积极加入了光荣的共产党。
    
朋友是人生路上的灯塔:迷茫时指引方向,困难时给予支持,成功时共同庆祝,快乐时分享喜悦,寂寞时陪伴左右。在南京大学,我结识了许多优秀的同学和朋友。我们相互成为了宝贵的财富。
    
家人是我的坚强后盾。幼时身体羸弱,父母外出打工时由祖父母照料成长。祖父母不辞辛劳地照顾我:祖父外出卖菜赚钱;祖母每天早上骑三轮车送我去医院看病再送我去上小学;初中时父母不遗余力地确保我能在县城里接受更好的教育;高中时他们安慰我的同时也默默流泪。从东南大学到南京大学,无论我多么任性或不懂事,在时间流逝中他们总是默默支持和鼓励我。

如果将百年人生比作一天,则青春如同早晨的太阳。在黎明之前,是家人、老师、朋友,以及党和国家在黑暗中给予的温暖和支持。当天亮之后,我愿成为那太阳,照亮自己的道路,温暖家人的生活,服务祖国,回馈社会,并为中国特色社会主义事业贡献自己的力量。

争做一个骄傲的社会主义事业的螺丝钉。
\end{acknowledgement}

\end{document}