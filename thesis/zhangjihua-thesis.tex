\documentclass[
    type = bachelor,
    degree = academic,
    twoside,
    fontset = win,
    decl-page
]
{njuthesis}

% njuthesis 参数设置文件 v1.4.0 2024-03-19

\njusetup[info]{
    title = {面向TLA\textsuperscript{+}规约的\\归纳不变式自动生成技术},
    title* = {Automatic Inductive Invariants Inference for  Specifications in TLA\textsuperscript{+} },
    author = {张继华},
    author* = {Jihua Zhang},
    keywords = {分布式协议, 形式化验证, 归纳不变式, TLA\textsuperscript{+}, 强化学习},
    keywords* = {Distributed Protocols, Formal Verification, Inductive Invariant, TLA\textsuperscript{+}, Reinforcement Learning},
    grade = {2020},
    student-id = {201250040},
    department = {软件学院},
    major = {软件工程},
    major* = {Software Engineering},
    supervisor = {魏恒峰,助理研究员},
    supervisor*= {Hengfeng Wei, Research Assistant},
    submit-date = {2024-05-27},
    % 提交日期
    supervisor-contact = {
        南京大学~
        江苏省南京市鼓楼区汉口路22号
    }
    % 导师联系方式
}

% bib 类用于参考文献设置
\njusetup[bib]{
    % style = numeric|author-year,
    % 参考文献样式
    % 默认为顺序编码制(numeric)
    % 可选著者-出版年制(author-year)
    %
    resource = {zhangjihua-thesis.bib},
    % 参考文献数据源
    % 需要带扩展名的完整文件名
    % 可使用逗号分隔多个文件
    % 此条等效于 \addbibresource 命令
    %
    % option = {
        % doi    = false,
        % isbn   = false,
        % url    = false,
        % eprint = false,
        % 关闭部分无用文献信息
        %
        % refsection = chapter,
        % 将参考文献表置于每章后
        %
        % gbnamefmt = lowercase
        % 使用仅首字母大写的姓名
    %   }
    % 额外的 biblatex 宏包选项
}

% image 类用于载入外置的图片
\njusetup[image]{
    % path = {{./figure/}{./image/}},
    % 图片搜索路径
    %
    nju-emblem = {nju-emblem},
    nju-name = {nju-name},
    % nju-emblem = {nju-emblem-purple},
    % nju-name = {nju-name-purple},
    % 替换为紫色版本
    % 这个选项只能填写一次
    % 切换时要注释掉上方的黑色版本
}

% abstract 类用于设置摘要样式
\njusetup[abstract]{
    toc-entry = true,
    % 摘要是否显示在目录条目中
    %
    % underline = false,
    % 研究生英文摘要页条目内容是否添加下划线
    %
    % title-style = strict|centered|natural
    % 研究生摘要标题样式,详见手册
}

% 目录自身是否显示在目录条目中
\njusetup{
    tableofcontents/toc-entry = false,
    % 关闭本项相当于同时关闭三个选项
    %
    % listoffigures/toc-entry   = false,
    % listoftables/toc-entry    = false
}

% 为目录中的章标题添加引导线
\njusetup[tableofcontents/dotline]{chapter}

% math 类用于设置数学符号样式,功能详见手册
\njusetup[math]{
    % style              = TeX|ISO|GB,
    % 整体风格,缺省值为国标(GB)
    % 相当于自动设置以下若干项
    %
    % integral           = upright|slanted,
    % integral-limits    = true|false,
    % less-than-or-equal = slanted|horizontal,
    % math-ellipsis      = centered|lower,
    % partial            = upright|italic,
    % real-part          = roman|fraktur,
    % vector             = boldfont|arrow,
    % uppercase-greek    = upright|italic
}

% theorem 类用于设置定理类环境样式,功能详见手册
\njusetup[theorem]{
    % define,
    % 默认创建内置的七种定理环境
    %
    % style         = remark,
    % header-font   = \sffamily \bfseries,
    % body-font     = \normalfont,
    % qed-symbol    = \ensuremath { \male },
    % counter       = section,
    % share-counter = true,
    % type          = {...}
    % 以上设置项在重新调用 theorem/define 后生效
}

% footnote 类用于设置脚注样式,功能详见手册
\njusetup[footnote]{
  % style = pifont|circled,
  % 使用圈码编号
  %
  % hang = false,
  % 不使用悬挂缩进
}

% 页眉页脚内容设置
\njusetup{
  % header/content = {
  %     {OR}{\thepage},{OL}{\rightmark},
  %     {EL}{\thepage},{ER}{\leftmark}
  %   },
  % 页眉设置,详见手册
  % 奇数页页眉:左侧章名,右侧页码
  % 偶数页页眉:左侧页码,右侧节名
  %
  % footer/content = {}
}

% 页眉页脚的字体样式
% \njusetformat{header}{\small\kaishu}
% \njusetformat{footer}{}

% 一些灵活调整
\njusetname{type}{本科毕业设计}                 % 我做的是毕业设计
% \njusetname{notation}{术语表}                   % 更改符号表名称
% \njusetlength{crulewd}{240pt}                   % 加长封面页下划线
% \njusetformat{tabular}{\zihao{-4}\bfseries}     % 修改表格环境的字号
% \EditInstance{nju}{u/cover/emblem-img}{align=l} % 左对齐的本科生封面校徽

\usepackage{listings} % 展示代码
\usepackage{algorithm,algorithmic} % 展示算法伪代码
\usepackage{subcaption} % 嵌套小幅图像,比 subfig 和 subfigure 更新更好
\usepackage{siunitx} % 标准单位符号

\newcommand{\TLA}{TLA\textsuperscript{+}\ }

\begin{document}
% 封面
\maketitle

\begin{abstract}
    分布式协议以及分布式系统,在当今的计算机世界不可或缺。自动化地对分布式协议验证其正确性是一个重要且困难的挑战。
    为了验证分布式协议的正确性,我们可以尝试证明分布式协议的每一个状态都满足一个预先给定的不变式,即安全属性(safety property)。
    对于复杂系统,我们无法像验证简单系统,通过遍历所有状态的方式来验证安全属性。
    过去的研究中往往采用寻找一个蕴含着安全属性的不变式的方式来验证协议的正确性,这个不变式经过所有可能的状态转移后仍然能保持其自身的正确性。
    这个不变式被称为归纳不变式。
    自动化地生成分布式协议的归纳不变式是验证自动化分布式协议正确性的关键步骤。

    以往的研究主要基于的IVy进行实现,\TLA 语言领域的相关研究较少。本文将提供一种基于\TLA 的归纳不变式生成工具,实现对以\TLA 语言描述的分布式协议规约的归纳不变式的自动化生成。
    与此同时,通过引入了强化学习的方法,加速归纳不变式的推导。
    引入外部工具 \href{https://apalache.informal.systems/}{apalache} 作为验证引擎,实现了对归纳不变式的验证。
    在进入强化学习模块之前,我们基于 tla2tools 对 \TLA 源文件进行静态分析,提取出语义信息,以便强化学习框架理解 \TLA 文件的语义。
    通过实验,本文提出的方法的有效性和可行性可以得到验证。
\end{abstract}
  
\begin{abstract*}
    Distributed Protocols and distributed systems are indispensable in today's computer world. Automatically verifying the correctness of distributed protocols is an important and challenging task.
    To verify the correctness of distributed protocols, we can try to prove that every state of the distributed protocol satisfies a pre-defined invariant, a safety property.
    For complex systems, we cannot verify safety properties by traversing all states as we do for simple systems.
    Researchers often use the method of finding an invariant that implies the safety property, which remains correct after all possible state transitions. This invariant is called an inductive invariant.
    Automatically generating inductive invariants for distributed protocols is a key step in verifying the correctness of automated distributed protocols.

    Previous research is mainly based on IVy for implementation, few on \TLA. 
    This paper will provide an inductive invariant generation tool based on \TLA, which realizes the automatic generation of inductive invariants for distributed protocol specifications described in \TLA language.
    Meanwhile, it accrelerates the derivation of inductive invariants by introducing reinforcement learning methods.
    The external tool \href{https://apalache.informal.systems/}{apalache} is introduced as a verification engine to verify the inductive invariants.
    Before entering the reinforcement learning module, we perform static analysis on the \TLA source files using tla2tools to extract semantic information for the reinforcement learning framework to understand the semantics of the \TLA files.
    Through experiments, the effectiveness and feasibility of the method proposed in this paper have been verified.
\end{abstract*}

% 目录
\tableofcontents

% 正文
\mainmatter
\chapter{绪论}

\section{研究背景和意义}
自动化地对分布式协议验证其正确性是一个重要且困难的挑战。
为了验证分布式协议的正确性,我们可以尝试证明分布式协议的每一个状态都满足一个预先给定的不变式,亦即安全属性(safety property)。
对于小型系统,我们可以通过遍历每个状态的方式进行验证。
然而,在工业实践中,分布式协议的参数众多且庞大,使得系统状态的数量巨大,我们无法简单地采用遍历的方式来验证安全属性是否在每个状态下都成立。
过去的研究中往往采用寻找一个蕴含着安全属性的不变式的方式来验证协议的正确性,这个不变式经过所有可能的状态转移后仍然能保持其自身的正确性。
这个不变式被称为归纳不变式。自动化地寻找或者生成分布式协议的归纳不变式是验证自动化分布式协议正确性的关键步骤。对于分布式系统,研究表明,给定一个正确的归纳不变式,几乎所有其他证明工作都可以自动完成。

\TLA \cite{TLA+}是一个对程序和系统,尤其对并发和分布式的程序和系统进行规约建模的高级语言。
在并发和分布式系统设计和开发过程中,非常容易发生基础性的设计问题,这些问题往往难以被发现。
而\TLA 以及其工具,利用集合论和时态逻辑精确地表达系统的状态和行为,可以帮助开发人员在设计阶段避免这些问题,以及在开发阶段定位问题。

目前,归纳不变式的自动生成技术,大多基于Ivy \cite{Ivy} 实现。
然而,Ivy的功能相比较\TLA 比较局限,且\TLA 在工业界的应用更加广泛。
当前针对\TLA 规约的自动化归纳不变式生成方法较少,且实现方法比较单一。
我们希望能\TLA 语言上开发出一种新的自动化归纳不变式生成方法,借助机器学习的技术以提高生成效率,为分布式协议的设计和验证提供帮助。

\section{研究问题}
本文使用 Python 和 Java 语言,实现对以\TLA 语言描述的分布式协议规约的归纳不变式的自动化生成工具 RLTLA,并通过实验对工具性能进行基准测试。
% TODO

\section{国内外研究现状}
目前的归纳不变式生成技术主要是基于 Ivy \cite{Ivy} 实现的,科研人员基于 Ivy 的平台设计了诸多归纳不变式自动生成的算法和工具,
基于 \TLA 的研究现在相对较少。

从实现理念和思路上,这些工具的大致可以分为两类,一种是基于程序语义(syntax-guided)的白盒技术,另一种是基于程序行为的黑盒技术。
近年来,随着 AI-for-SE的发展,一种叫做ICE\cite{ICE}(implication counterexamples)的学习框架流行起来,它将不变式的证明工作分为了两个部分:学习者和教育者。
依赖随机搜索、决策树\cite{garg2016learning}、强化学习\cite{LIPuS}等技术,许多工作推进了学习者模块的发展。
此外,也有人将新颖的语言大模型引入了不变式生成的工作中\cite{llm}。
白盒技术和黑盒技术的界限并不明确,一些工具其实兼而有之地采取两种技术的优势。

DistAI\cite{DistAI}以及DuoAI\cite{DuoAI}来自同一个研究团队,使用枚举候选不变式的算法进行自动不变式生成。
他们基于已有的小体积的运行数据,在削减过的空间上,在有限的句法空间中通过工具裁剪谓词来生成候选不变式,
也就是说对已有的运行数据,依赖协议中已有的谓词进行抽象,总结出候选不变式。然后对候选不变式进行验证并对失败的候选不变式继续裁剪,
最后按照一定顺序进行枚举,组合成为最终的归纳不变式;
I4\cite{I4}基于有限实例推广进行自动不变式生成。
I4会首先创建协议的多个有限实例,利用模型检查工具自动导出该有限实例的归纳不变式。
然后,I4尝试进行泛化,将不变式推广到更大的实例上,最终使之在协议上成立;
LIPuS 则在基于语义的基础上,使用了强化学习的框架对搜索空间进行剪枝,并在修剪过后的空间上进行 SMT 求解。使用这种方式可以有效地减少对SMT solver 的调用,从而提高生成归纳不变式的效率。

以上的工作,均是基于 Ivy 的。目前基于 \TLA 的归纳不变式生成工具较少。

endive\cite{endive}基于I4的思想,是第一个基于 \TLA 的归纳不变式生成工具。
endive 需要用户提供原子公式,算法就会根据这些原子公式自动生成可能的归纳不变式,并且从中排除掉那些违反安全属性的不变式。
之后,endive 会选择可以杀死反例最多的不变式,并重复这一过程,直到组合出最终的归纳不变式。

\section{本文组织结构}
本文的组织结构如下:


\chapter{预备知识}\label{chap:pre-knowleage}

本章节将以规约 \textit{Client\_Server} (图\ref{fig:client_server})为例,
介绍 \TLA 规约的基本结构,以及在寻找归纳不变式过程中的其他预备知识。
\begin{figure}
    \centering
    \includegraphics[width=0.8\textwidth]{figures/Client_Server.pdf}
    \caption{Client\_Server 规约}
    \label{fig:client_server}
\end{figure}

\section{\texorpdfstring{TLA\textsuperscript{+}}{TLA+}}
\href{https://lamport.azurewebsites.net/tla/tla.html}{\TLA} \cite{TLA}是由计算机科学家 Leslie Lamport 主导开发的,
用于对计算机程序和系统建模,尤其是对并行系统和分布式系统建模的高级语言。
它是基于使用简单的数学语言来精确描述系统行为的理念开发的。
因此,\TLA 的表达方式和一般的编程语言有很大的不同,反而和数学语言更为接近。
\TLA 并不是一种编程语言,而是一种规约语言,它不关注协议或者系统的具体实现,从而能更高层次看到程序整体的设计。
因此,\TLA 及其工具对于消除代码中很难发现和纠错成本高昂的错误非常有用。

需要注意到的是,\TLA 并不是为了寻找归纳不变式而设计的,而是为了对系统进行建模,是为了让规约开发人员能更好地表达一个协议。
它的语法更加丰富,以更加直观的方式表达一个协议。

开发者使用 \TLA 或者其他工具来对分布式协议进行建模的代码,我们将其称之为规约(specification,简称spec)。
图 \ref{fig:client_server} 展示了一个简单的 \TLA 规约,其中包含了一个简单的客户端和服务器的通信协议。
其中两个重要的谓词是 $Init$ 和 $Next$。
$Init$ 表示系统的初始状态,描述系统最开始时的状态;
而$Next$ 则是表示系统的状态是如何转移,也就是系统的状态在每个时间片后会发生怎样的变化。
在这个规约中,其他的谓词还有$Connect, Disconnect$ 等,定义这些谓词,就像在一般的编程语言中定义函数一样,方便阅读和重复使用。
一些规约中还有谓词 $TypeOK$,用于约束变量的类型。
另一种在自动归纳不变式生成研究中常常使用的工具,IVy,也有相似的语法和结构。
可以看到的是,\TLA 更关注系统的状态和系统状态是发生怎样的转移,对于系统状态转移的具体实现,\TLA 并不关心。
这样的描述方式和状态机非常相似。

谓词$Safe$ 是安全属性(safety property),一个正确定义的分布式协议规约,应当在每个可达的状态下都满足安全属性。
这个变量在自动化生成归纳不变式的研究中非常关键。

TLC 是 \TLA 集成的模型检测工具。
除了 TLC 以外,\TLA toolbox\cite{tla+toolbox} 还集成有PlusCal\cite{PlusCal}和TLAPS用于命题证明工具,sany用于语法检查工具,
tex 用于将\TLA 美化打印的工具等,这些工具与本文所讨论的问题相关性不高,不展开讨论。

本文所述工具接受\TLA 的规约。

\section{归纳不变式和归纳反例}
验证分布式协议的正确性,就是验证协议定义的安全属性(safety property)是否在每个可达的状态下都成立。
在 \textit{Client\_Server} 规约中,我们可以看到 $Safe$ 是一个安全属性,
它表达的是,在任何状态下,如果两个客户端同时连接有同一个服务器,那么这两个客户端是同一个客户端。
换言之,两个不同的客户端不能连接到同一个服务器。

对于简单的系统,即变量和状态不多的系统,我们可以通过遍历每一个可能的状态来验证。
但是对于稍微复杂一些的系统,尤其是越来越多的分布式系统,规模越来越大,状态也越来越复杂。
通过简单的遍历的方式来验证系统的正确性,是不现实的。
寻找一个能够蕴含安全属性的不变式,并且能够在所有可能的状态转移后保持其自身的正确性,这个不变式被称为归纳不变式。
以数学的语言表示为:
\begin{align}
    &Init \Rightarrow Ind \label{con:init}\\
    &Ind \land Next \Rightarrow Ind' \label{con:inductive}\\
    &Ind \Rightarrow Safe \label{con:safety}
\end{align}
其中$Init$ 表示初始状态,$Next$ 表示状态转移,$Safe$ 是安全属性,$Ind$ 表达的是归纳不变式,
而$Ind'$表达谓词$Ind$经过状态转移后的变量的状态。
定理\ref{con:init}表明归纳不变式在初始状态下成立;
定理\ref{con:inductive}表明归纳不变式在状态转移后依然成立,具有归纳性质。
比如说,如果$Ind$在状态$s$下成立,那么在$s$的后继状态下,$Ind$依然成立;
定理\ref{con:safety}表明归纳不变式蕴含安全属性,因此,如果某个运行时可达状态满足$Ind$, 那么也必然满足$Safe$。
这是我们寻找一个这样的归纳不变式的目的,通过归纳不变式的正确性验证安全属性的正确性。
这是归纳不变式所必须满足的三个条件。

\begin{align}
    &\left.A_{1} \triangleq \forall s \in  { Server }: \forall c \in  { Client }:  { locked }[s] \Rightarrow(s \notin { held }[c])\right) \\
    &{ Ind } \triangleq  { Safe } \wedge A_{1} \label{con:candidate_ind}
\end{align}

对于\textit{Client\_Server} 规约,表达式\ref{con:candidate_ind}是一个可能的归纳不变式。
可以看到的是,$Ind$是由$Safe$和$A_{1}$两个谓词逻辑表达式合取组成而来。
事实上,大部分规约的归纳不变式都可以表达为$Ind \triangleq Safe \wedge A_1 \wedge A_2 \wedge... \wedge A_n$的形式。
其中析取子式$A_k$是约束状态的谓词,我们将之称为引理不变式(Lemma Invariant)。
因为归纳不变式$Ind$是由这些引理不变式$A_k$组合而成的,也就是说,归纳不变式强于每一个引理不变式。
因此,引理不变式需要满足不变性,也就是在系统运行的每个状态下都成立,才能成为一个合适的引理不变式。
但是,引理不变式本身不需要满足归纳性,只需要他们和安全属性的析取结果能够满足归纳性。

对于一个谓词表达式$P$,如果一个状态$s$满足$s \models P$,但是$s$的后继状态$s_{n} \models \neg P$,
那便可以称$s$ 为 $P$的归纳反例(counterexample),揭示了$P$不是归纳不变式。
一个归纳反例往往包括两个状态,前一个状态满足谓词$P$,而后一个状态不满足谓词$P$。

\begin{figure}
    \centering
    \includegraphics[width=0.7\textwidth]{figures/ind-cti.pdf}
    \caption{归纳不变式和归纳反例}
    \label{fig:ind-cti}
\end{figure}
图\ref{fig:ind-cti}形象地介绍了归纳不变式和归纳反例。
寻找归纳不变式的过程,也可以理解为通过添加新的引理不变式,来排除归纳反例的过程。
但是,尤其是对于越复杂的系统而言,寻找归纳不变式并不是一个简单的任务。
实现归纳不变式的自动生成是形式化验证领域一个重要的研究目标,这也是本文研究的内容。

\section{TLC 和 Apalache}
TLC和Apalache是两个常见的面向\TLA 规约的模型检查工具(model checker)。
本文通过TLC和Apalache来验证生成的候选不变式的性质。
主要是借助TLC或Apalache获取归纳反例,以及验证生成模块生成的谓词表达式是否是有限实例上的不变式。

\subsection{TLC}
TLC既是对\TLA 规约的模型检查工具,也是一个面向规约的模拟器。
它是一个显式状态模型检查器,依照用户给出的规约和设置,搜索所有满足约束的状态和状态转移,
并在这个过程中检查安全属性和其他用户定义的谓词逻辑时时是否成立。
如果遇到错误,TLC会将错误的状态和状态转移过程输出,以便用户进行分析。

TLC可以通过使用超过32个计算机线程以获得近乎线性的加速。
它可以通过在分布式部署的计算机网络上运行来进一步加速模型检查,并提供在云系统上的轻松部署。

\subsection{Apalache}
Apalache\cite{apalache1, apalache2} 和 TLC 不同的是,Apalache 并不是通过遍历所有可能的状态来检验安全属性是否成立,而是通过 SMT solver 来检验。
它是将\TLA 规约转换为 SMT 问题,然后使用 SMT solver (如Z3\cite{z3})求解来检验安全属性是否成立。
Apalaches 是一种符号检查器,它和 SMT solver 一样基于逻辑推理和公式求解实现的。
Apalache 对\TLA 源文件的语法中引入了一些限制,
尽管没有完全支持\TLA 的所有语法,但是这方便使用 SMT 求解器进行求解。

\TLA 是一个“弱类型”的编程语言,它对变量没有严格的类型注明。
但是,Apalache 需要了解\TLA 规约中变量的类型才能工作。
尽管 Apalache 有一套自己的类型推断系统,但是,它并不能完全解决所有的类型推断问题。
这使得用户,对于某些协议,需要以注释的形式来提供变量的类型,才能交给Apalache进行处理。

% 补充关于 返回结果和CTI的内容?

\section{强化学习}
强化学习(Reinforcement learning, RL)\cite{rl}是机器学习的一个领域,强调如何基于外部环境做出决策,以获得最大化的预期累积奖励。
是区别于监督学习和非监督学习的另外一种基本的机器学习方法。
强化学习的关注点在于寻找对未知领域的探索和对已有知识的利用之间的平衡。
它的目标是通过奖惩来控制智能体完成任务,以获得最大化的预期累积奖励,但程序无需明确告诉智能体如何完成任务。

在机器学习问题中,环境通常被抽象为马尔可夫决策过程(Markov decision processes,MDP)\cite{markov},
因为很多强化学习算法在这种假设下才能使用动态规划的方法。
传统的动态规划方法和强化学习算法的主要区别是,后者不需要关于MDP的知识,而且针对无法找到确切方法的大规模MDP。

\begin{figure}[h]
    \centering
    \includegraphics[width=0.6\textwidth]{figures/Reinforcement_learning_diagram.pdf}
    \caption{强化学习框架}
    \label{fig:rl}
\end{figure}
图 \ref{fig:rl} 展示了强化学习的框架。
强化学习中,主要是智能体(agent)和环境(environment)之间的交互。
环境是智能体所处的环境,它会根据智能体的动作,给予智能体奖励或者惩罚,并作出状态转移。
智能体根据奖励和环境状态的变化来调整自己的策略。
智能体,可以感知环境的状态(State),并根据反馈的奖励(Reward)学习选择一个合适的动作(Action),来最大化长期总收益。
智能体会根据环境的反馈,调整自己的策略,以获得最大化的预期累积奖励。
实现强化学习的策略算法有很多,其中最著名的有 Deep Q Network(DQN)\cite{dqn},Q-learning\cite{q-learning}等。

在本文的项目中,我们使用强化学习的方法来加速归纳不变式的生成。
我们使智能体理解\TLA 原文件的内容,让智能体合理选择生成归纳不变式的种子(seed),
并将每一次智能体选择的种子所生成的不变式的检验结果反馈给智能体,包括反例的数量,内容和生成时间等。
智能体根据这些反馈信息,调整自己的策略,以便更快地找到一个满足安全属性的归纳不变式。



\chapter{总结与分析}
本章对论文工作进行了总结,并展望了未来可能的优化和改进方向。
\section{工作总结}
本文主要目的是验证机器学习,尤其是强化学习在对分布式系统规约的归纳不变式生成领域中的可行性和有效性。

本文在\TLA 的语言平台上,实现了一个基于强化学习的归纳不变式生成系统,
介绍了系统设计的预备知识和理论基础,以RLTLA的系统体系结构的设计和实现。

\TLA 相较于 IVy 更加复杂,其中存在灵活多样的数据结构,同时也支持任意的嵌套来表达规约。
这对开发人员在设计阶段表达系统的行为和状态转移关系提供了很大的便利,但这点对于归纳不变式生成工具的设计并不友好。

实现上,本文依靠于\TLA 源文件和 endive 对于\TLA 解析和人工识别的假设作为输入,基于 try-and-error 的生成思路,
采用强化学习的方式生成候选不变式,通过模型检查器验证候选不变式的正确性,独立性以及当前所有候选不变式析取结果的递归性,最终生成归纳不变式。
这一生成思路和大部分的归纳不变式生成工具类似,但是在实现上,本文寄希望于强化学习以提高生成效率。

本文使用 gymnasium 实现强化学习的算法,并借助 tianshou 提供的强化学习算法,实现完整的强化学习模块。
强化学习模块接受TLC或Apalache的反馈,基于已经给出的假设和\TLA 源文件生成候选不变式,并交给模型检查器验证。
gymnasium 提供了标准的环境接口,是目前十分受欢迎的环境开源工作,方便了强化学习环境的实现,并且可以接入多种强化学习算法。
tianshou 是一个完全基于 Pytorch 的强化学习框架,方便我们使用多种算法对强化学习模块进行训练和比较。

本文使用TLC和Apalache对生成的候选不变式进行验证,检验生成的候选不变式的正确性,独立性和与已有不变式析取结果的递归性,并将结果返回给强化学习模块。
TLC和Apalache没有提供python的接口,本文通过调用命令行的方式调用TLC和Apalache,并通过对命令行结果的解析,获取验证结果。

在测试部分,本文使用 endive 提供的测试用例对系统进行测试,并和endive进行了比较,
验证了强化学习在面向分布式系统规约的归纳不变式生成工作中具有可行性和有效性。

\section{未来展望}
由于时间和能力的限制,本文所实现的系统的性能和实现方式上有许多不足,存在大量的改进空间。

目前,和endive一样,RLTLA还以来于一些人工的输入,即一些人工识别的假设(predicates)
,这些假设的得到是依赖于人脑对于 \TLA 规约理解,尤其是对每个 Action 进行调用时参数类型的理解。
如果要自动化地识别和生成这些假设,也是一个复杂的工作,目前系统还不具备这项能力。
与此同时,人脑的参与可能带来效率的下降和不可预计的错处的出现。
未来需要实现一个功能更加丰富的静态分析工具,以自动提取出 \TLA 规约的语义信息,包括 Action 的参数类型等,
以帮助强化学习模块更好地理解 \TLA 规约和生成合适的候选不变式,或者通过接入大模型的方式,完成归纳不变式的生成。
另一方面,人工的输入也约束了可搜索的空间的大小,在这一条件下,
如同 endive 的通过机械搜索的方式,可能获得比强化学习更快的生成效率。

其次,目前提供给系统可以选择的谓词的范围基于\TLA 已经定义的最高层次的谓词,并没有考虑谓词的子式以及谓词之间的关系。
系统也无法考虑给出的这些谓词表达式之间,以及每个可能的候选不变式之间的关系。
目前系统设计上,一次只生成一个可能的候选不变式,
这两方面因素,导致了对于每一次的生成的候选不变式的检查,都需要调用1-3次模型检查器进行检查,这往往很花时间,导致系统的效率较低。

基于\TLA 的归纳不变式的生成是一个复杂的问题,在这一领域的研究不是十分充足,可以参考和对比的工作较少。
目前,对于基于\TLA 的归纳不变式生成工具还没有统一的测试集合,也没有十分充足的测试用例。
这导致我们一方面很难评估系统的效率,另一方面,也很难提供给强化学习模块足够的训练数据。
在 endive 的测试集合下,一个归纳不变式常常只需要不超过10个子式析取而来。
在这一背景下,强化学习的效率并不理想,常常带来相较于 endive 等工作提供的搜索算法更高的开支和更低的效率。



% 生成参考文献页
\printbibliography

\begin{acknowledgement}
时光荏苒,四年的时间很快过去了。2020年9月3日大清早,拖着行李走进南大校门,穿过教学楼和一组团主干道的场景还历历在目,仿佛就在昨天,如今却快要到达四年本科学习的终点,走向下一段旅程。
本科四年,求学生涯,乃至人生到此的所有阶段,都离不开许多人的帮助和支持。

感谢父母,祖父母对我的养育之恩,对我生活和求学的支持。是他们的付出,让我有成长学习的环境。

感谢魏恒峰老师在学习生活上的指导。是他带领我进入科研的大门,让我了解分布式系统验证和形式化方法。

感谢诸位朋友的陪伴,开导,帮助和建议,人生因为你们而丰富多彩。

感谢党和国家对教育事业的支持。感谢南京大学,以及诸位老师,工作人员为我们提供了一个良好的学习环境。

\end{acknowledgement}

\end{document}