\documentclass[
    type = bachelor,
    degree = academic,
    twoside,
    fontset = win,
    decl-page
]
{njuthesis}

% njuthesis 参数设置文件 v1.4.0 2024-03-19

\njusetup[info]{
    title = {面向TLA\textsuperscript{+}规约的\\归纳不变式自动生成技术},
    title* = {Automatic Inductive Invariants Learning for TLA\textsuperscript{+} Specifications},
    author = {张继华},
    author* = {Jihua Zhang},
    keywords = {TLA\textsuperscript{+}, 分布式系统, 形式化验证, 强化学习, 归纳不变式},
    keywords* = {TLA\textsuperscript{+}, Distributed System, Formal Verification, Reinforcement Learning, Inductive Invariant},
    grade = {2020},
    student-id = {201250040},
    department = {软件学院},
    major = {软件工程},
    major* = {Software Engineering},
    supervisor = {魏恒峰,助理研究员},
    supervisor*= {Hengfeng Wei, Research Assistant},
    submit-date = {2024-05-20},
    % 提交日期
    supervisor-contact = {
        南京大学~
        江苏省南京市鼓楼区汉口路22号
    }
    % 导师联系方式
}

% bib 类用于参考文献设置
\njusetup[bib]{
    % style = numeric|author-year,
    % 参考文献样式
    % 默认为顺序编码制(numeric)
    % 可选著者-出版年制(author-year)
    %
    resource = {zhangjihua-thesis.bib},
    % 参考文献数据源
    % 需要带扩展名的完整文件名
    % 可使用逗号分隔多个文件
    % 此条等效于 \addbibresource 命令
    %
    % option = {
        % doi    = false,
        % isbn   = false,
        % url    = false,
        % eprint = false,
        % 关闭部分无用文献信息
        %
        % refsection = chapter,
        % 将参考文献表置于每章后
        %
        % gbnamefmt = lowercase
        % 使用仅首字母大写的姓名
    %   }
    % 额外的 biblatex 宏包选项
}

% image 类用于载入外置的图片
\njusetup[image]{
    % path = {{./figure/}{./image/}},
    % 图片搜索路径
    %
    nju-emblem = {nju-emblem},
    nju-name = {nju-name},
    % nju-emblem = {nju-emblem-purple},
    % nju-name = {nju-name-purple},
    % 替换为紫色版本
    % 这个选项只能填写一次
    % 切换时要注释掉上方的黑色版本
}

% abstract 类用于设置摘要样式
\njusetup[abstract]{
    toc-entry = true,
    % 摘要是否显示在目录条目中
    %
    % underline = false,
    % 研究生英文摘要页条目内容是否添加下划线
    %
    % title-style = strict|centered|natural
    % 研究生摘要标题样式,详见手册
}

% 目录自身是否显示在目录条目中
\njusetup{
    tableofcontents/toc-entry = false,
    % 关闭本项相当于同时关闭三个选项
    %
    % listoffigures/toc-entry   = false,
    % listoftables/toc-entry    = false
}

% 为目录中的章标题添加引导线
\njusetup[tableofcontents/dotline]{chapter}

% math 类用于设置数学符号样式,功能详见手册
\njusetup[math]{
    % style              = TeX|ISO|GB,
    % 整体风格,缺省值为国标(GB)
    % 相当于自动设置以下若干项
    %
    % integral           = upright|slanted,
    % integral-limits    = true|false,
    % less-than-or-equal = slanted|horizontal,
    % math-ellipsis      = centered|lower,
    % partial            = upright|italic,
    % real-part          = roman|fraktur,
    % vector             = boldfont|arrow,
    % uppercase-greek    = upright|italic
}

% theorem 类用于设置定理类环境样式,功能详见手册
\njusetup[theorem]{
    % define,
    % 默认创建内置的七种定理环境
    %
    % style         = remark,
    % header-font   = \sffamily \bfseries,
    % body-font     = \normalfont,
    % qed-symbol    = \ensuremath { \male },
    % counter       = section,
    % share-counter = true,
    % type          = {...}
    % 以上设置项在重新调用 theorem/define 后生效
}

% footnote 类用于设置脚注样式,功能详见手册
\njusetup[footnote]{
  % style = pifont|circled,
  % 使用圈码编号
  %
  % hang = false,
  % 不使用悬挂缩进
}

% 页眉页脚内容设置
\njusetup{
  % header/content = {
  %     {OR}{\thepage},{OL}{\rightmark},
  %     {EL}{\thepage},{ER}{\leftmark}
  %   },
  % 页眉设置,详见手册
  % 奇数页页眉:左侧章名,右侧页码
  % 偶数页页眉:左侧页码,右侧节名
  %
  % footer/content = {}
}

% 页眉页脚的字体样式
% \njusetformat{header}{\small\kaishu}
% \njusetformat{footer}{}

% 一些灵活调整
\njusetname{type}{本科毕业设计}                 % 我做的是毕业设计
% \njusetname{notation}{术语表}                   % 更改符号表名称
% \njusetlength{crulewd}{240pt}                   % 加长封面页下划线
% \njusetformat{tabular}{\zihao{-4}\bfseries}     % 修改表格环境的字号
% \EditInstance{nju}{u/cover/emblem-img}{align=l} % 左对齐的本科生封面校徽

\usepackage{listings} % 展示代码
\usepackage{algorithm,algorithmic} % 展示算法伪代码
\usepackage{subcaption} % 嵌套小幅图像,比 subfig 和 subfigure 更新更好
\usepackage{siunitx} % 标准单位符号

\newcommand{\TLA}{TLA\textsuperscript{+}\ }

\begin{document}
% 封面
\maketitle

\begin{abstract}
    分布式协议以及分布式系统,在当今的计算机世界不可或缺。自动化地对分布式协议验证其正确性是一个重要且困难的挑战。
    为了验证分布式协议的正确性,我们可以尝试证明分布式协议的每一个状态都满足一个预先给定的不变式,即安全属性(safety property)。
    对于复杂系统,我们无法像验证简单系统,通过遍历所有状态的方式来验证安全属性。
    过去的研究中往往采用寻找一个蕴含着安全属性的不变式的方式来验证协议的正确性,这个不变式经过所有可能的状态转移后仍然能保持其自身的正确性。这个不变式被称为归纳不变式。
    自动化地寻找或者生成分布式协议的归纳不变式是验证自动化分布式协议正确性的关键步骤。

    以往的研究主要基于的IVy进行实现,\TLA 语言领域的相关研究较少。本文将提供一种基于\TLA 的归纳不变式生成工具,实现对以\TLA 语言描述的分布式协议规约的归纳不变式的自动化生成。
\end{abstract}
  
\begin{abstract*}
    Distributed Protocols and distributed systems are indispensable in today's computer world. Automatically verifying the correctness of distributed protocols is an important and challenging task.
    To verify the correctness of distributed protocols, we can try to prove that every state of the distributed protocol satisfies a pre-defined invariant, a safety property.
    For complex systems, we cannot verify safety properties by traversing all states as we do for simple systems.
    Researchers often use the method of finding an invariant that implies the safety property, which remains correct after all possible state transitions. This invariant is called an inductive invariant.
    Automatically finding or generating inductive invariants for distributed protocols is a key step in verifying the correctness of automated distributed protocols.

    Previous research is mainly based on IVy for implementation, few on \TLA. 
    This paper will provide an inductive invariant generation tool based on \TLA, which realizes the automatic generation of inductive invariants for distributed protocol specifications described in \TLA language.
    
\end{abstract*}

% 目录
\tableofcontents

% 正文
\mainmatter

% 生成参考文献页
\printbibliography

\begin{acknowledgement}
    致谢
\end{acknowledgement}

\end{document}