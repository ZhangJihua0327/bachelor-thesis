\chapter{总结与分析}
本章对论文工作进行了总结,并且展望了未来可能的优化方向。
\section{工作总结}
工作总结
\section{未来展望}
由于时间和能力的限制,本文所实现的系统的性能和实现方式上还有许多优化空间。我们认为未来可能的工作如下:
\begin{enumerate}
    \item 当前,系统的输入还依赖于一些人工识别的假设,这些假设的得到是依赖于人脑对于 \TLA 规约理解,
    尤其是对每个 Action 进行调用时参数类型的理解。人脑的参与可能带来效率的下降和不可预计的错处的出现。
    未来需要实现一个功能更加丰富的静态分析工具,以自动提取出 \TLA 规约的语义信息,包括 Action 的参数类型等,
    以帮助强化学习模块更好地理解 \TLA 规约和生成合适的候选不变式;
    \item 目前,系统可以选择的谓词的范围基于\TLA 已经定义的最高层次的谓词,并没有考虑谓词的子式以及谓词之间的关系。
    后续的工作中,可以考虑将\TLA 规约中的谓词进行划分,并将谓词以图的方式输入到系统中;
    \item 目前的测试用例局限于endive 中所提供的,相较于 IVy 相关的研究,测试用例的数量还是较少,
    并且没有形成统一的测试集。
\end{enumerate}