\chapter{绪论}

\section{研究背景和意义}
自动化地对分布式协议验证其正确性是一个重要且困难的挑战。
为了验证分布式协议的正确性,我们可以尝试证明分布式协议的每一个状态都满足一个预先给定的不变式,亦即安全属性(safety property)。
对于小型系统,我们可以通过遍历每个状态的方式进行验证。
然而,在工业实践中,分布式协议的参数众多且庞大,使得系统状态的数量巨大,我们无法简单地采用遍历的方式来验证安全属性是否在每个状态下都成立。
过去的研究中往往采用寻找一个蕴含着安全属性的不变式的方式来验证协议的正确性,这个不变式经过所有可能的状态转移后仍然能保持其自身的正确性。
这个不变式被称为归纳不变式。自动化地寻找或者生成分布式协议的归纳不变式是验证自动化分布式协议正确性的关键步骤。对于分布式系统,研究表明,给定一个正确的归纳不变式,几乎所有其他证明工作都可以自动完成。

\TLA \cite{TLA+}是一个对程序和系统,尤其对并发和分布式的程序和系统进行规约建模的高级语言。
在并发和分布式系统设计和开发过程中,非常容易发生基础性的设计问题,这些问题往往难以被发现。
而\TLA 以及其工具,利用集合论和时态逻辑精确地表达系统的状态和行为,可以帮助开发人员在设计阶段避免这些问题,以及在开发阶段定位问题。

目前,归纳不变式的自动生成技术,大多基于Ivy \cite{Ivy} 实现。
然而,Ivy的功能相比较\TLA 比较局限,且\TLA 在工业界的应用更加广泛。
当前针对\TLA 规约的自动化归纳不变式生成方法较少,且实现方法比较单一。
我们希望能\TLA 语言上开发出一种新的自动化归纳不变式生成方法,借助机器学习的技术以提高生成效率,为分布式协议的设计和验证提供帮助。

\section{研究问题}
本文使用 Python 和 Java 语言,实现对以\TLA 语言描述的分布式协议规约的归纳不变式的自动化生成工具 RLTLA,并通过实验对工具性能进行基准测试。
% TODO

\section{国内外研究现状}
目前的归纳不变式生成技术主要是基于 Ivy \cite{Ivy} 实现的,科研人员基于 Ivy 的平台设计了诸多归纳不变式自动生成的算法和工具,
基于 \TLA 的研究现在相对较少。

从实现理念和思路上,这些工具的大致可以分为两类,一种是基于程序语义(syntax-guided)的白盒技术,另一种是基于程序行为的黑盒技术。
近年来,随着 AI-for-SE的发展,一种叫做ICE\cite{ICE}(implication counterexamples)的学习框架流行起来,它将不变式的证明工作分为了两个部分:学习者和教育者。
依赖随机搜索、决策树\cite{garg2016learning}、强化学习\cite{LIPuS}等技术,许多工作推进了学习者模块的发展。
此外,也有人将新颖的语言大模型引入了不变式生成的工作中\cite{llm}。
白盒技术和黑盒技术的界限并不明确,一些工具其实兼而有之地采取两种技术的优势。

DistAI\cite{DistAI}以及DuoAI\cite{DuoAI}来自同一个研究团队,使用枚举候选不变式的算法进行自动不变式生成。
他们基于已有的小体积的运行数据,在削减过的空间上,在有限的句法空间中通过工具裁剪谓词来生成候选不变式,
也就是说对已有的运行数据,依赖协议中已有的谓词进行抽象,总结出候选不变式。然后对候选不变式进行验证并对失败的候选不变式继续裁剪,
最后按照一定顺序进行枚举,组合成为最终的归纳不变式;
I4\cite{I4}基于有限实例推广进行自动不变式生成。
I4会首先创建协议的多个有限实例,利用模型检查工具自动导出该有限实例的归纳不变式。
然后,I4尝试进行泛化,将不变式推广到更大的实例上,最终使之在协议上成立;
LIPuS 则在基于语义的基础上,使用了强化学习的框架对搜索空间进行剪枝,并在修剪过后的空间上进行 SMT 求解。使用这种方式可以有效地减少对SMT solver 的调用,从而提高生成归纳不变式的效率。

以上的工作,均是基于 Ivy 的。目前基于 \TLA 的归纳不变式生成工具较少。

endive\cite{endive}基于I4的思想,是第一个基于 \TLA 的归纳不变式生成工具。
endive 需要用户提供原子公式,算法就会根据这些原子公式自动生成可能的归纳不变式,并且从中排除掉那些违反安全属性的不变式。
之后,endive 会选择可以杀死反例最多的不变式,并重复这一过程,直到组合出最终的归纳不变式。

\section{本文组织结构}
本文的组织结构如下:
