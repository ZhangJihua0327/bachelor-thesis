\chapter{预备知识}

\section{\texorpdfstring{TLA\textsuperscript{+}}{TLA+}}
\href{https://lamport.azurewebsites.net/tla/tla.html}{\TLA} \cite{TLA, tla+toolbox}是由计算机科学家 Leslie Lamport 主导开发的,
用于对计算机程序和系统建模,尤其是对并行系统和分布式系统建模的高级语言。
它是基于使用简单的数学语言来精确描述系统行为的理念开发的。
因此,\TLA 的表达方式和一般的编程语言有很大的不同,反而和数学语言更为接近。
\TLA 并不是一种编程语言,而是一种规约语言,它不关注协议或者系统的具体实现,从而能更高层次看到程序整体的设计。
因此,\TLA 及其工具对于消除代码中很难发现和纠错成本高昂的错误非常有用。

需要注意到的是,\TLA 并不是为了寻找归纳不变式而设计的,而是为了对系统进行建模,是为了让规约开发人员能更好地表达一个协议。
它的语法更加丰富,以更加直观的方式表达一个协议。



开发者使用 \TLA 或者其他工具来对分布式协议进行建模的代码,我们将其称之为规约(specification,简称spec)。

\section{归纳不变式}
验证分布式协议的正确性,往往不是一件简单的事情。
验证分布式协议的正确性,就是验证协议定义的安全属性(safety property)是否在每个可达的状态下都成立。
对于简单的系统,即变量和状态不多的系统,我们可以通过遍历每一个可能的状态来验证。
但是对于稍微复杂一些的系统,尤其是越来越多的分布式系统,规模越来越大,状态也越来越复杂。
通过简单的遍历的方式来验证系统的正确性,是不现实的。
寻找一个能够蕴含安全属性的不变式,并且能够在所有可能的状态转移后保持其自身的正确性,这个不变式被称为归纳不变式。
以数学的语言表示为:
\begin{equation}
\begin{aligned}
    &Init \Rightarrow Ind \\
    &Ind \land Next \Rightarrow Ind'\\
    &Ind \Rightarrow Safety
\end{aligned}
\end{equation}
其中 $Init$ 是 


\section{强化学习}