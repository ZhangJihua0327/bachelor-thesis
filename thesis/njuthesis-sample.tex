%%%%%%%%%%%%%%%%%%%%%%%%%%%%%%%%%%%%%%%%%%%%%%%%%%%%%%%%%%%%%%%%%%%%%%
% njuthesis 示例模板 v1.4.0 2024-03-19
% https://github.com/nju-lug/NJUThesis
%
% 贡献者
% Yu XIONG @atxy-blip   Yichen ZHAO @FengChendian
% Song GAO @myandeg     Chang MA @glatavento
% Yilun SUN @HermitSun  Yinfeng LIN @linyinfeng
%
% 许可证
% LaTeX Project Public License(版本 1.3c 或更高)
%%%%%%%%%%%%%%%%%%%%%%%%%%%%%%%%%%%%%%%%%%%%%%%%%%%%%%%%%%%%%%%%%%%%%%

%---------------------------------------------------------------------
% 一些提升使用体验的小技巧:
%   1. 请务必使用 UTF-8 编码编写和保存本文档
%   2. 请务必使用 XeLaTeX 或 LuaLaTeX 引擎进行编译
%   3. 不保证接口稳定,写作前一定要留意版本号
%   4. 以百分号(%)开头的内容为注释,可以随意删除
%---------------------------------------------------------------------

%---------------------------------------------------------------------
% 请先阅读使用手册:
% http://mirrors.ctan.org/macros/unicodetex/latex/njuthesis/njuthesis.pdf
%---------------------------------------------------------------------

\documentclass[
    % 模板选项:
    %
    % type = bachelor|master|doctor|postdoc, % 文档类型,默认为本科生
    % degree = academic|professional,        % 学位类型,默认为学术型
    %
    % nl-cover,   % 是否需要国家图书馆封面,默认关闭
    % decl-page,  % 是否需要诚信承诺书或原创性声明,默认关闭
    %
    %   页面模式,详见手册说明
    % draft,                  % 开启草稿模式
    % anonymous,              % 开启盲审模式
    % minimal,                % 开启最小化模式
    %
    %   单双面模式,默认为适合印刷的双面模式
    % oneside,                % 单面模式,无空白页
    % twoside,                % 双面模式,每一章从奇数页开始
    %
    %   字体设置,不填写则自动调用系统预装字体,详见手册
    % fontset = win|mac|macoffice|fandol|none,
  ]{njuthesis}

% 模板选项设置,包括个人信息、外观样式等
% 较为冗长且一般不需要反复修改,我们把它放在单独的文件里
% njuthesis 参数设置文件 v1.4.0 2024-03-19

% 一些提醒:
%   1. \njusetup 内部千万不要有空行
%   2. 使用英文半角逗号(,)分隔选项
%   3. 等于号(=)两侧的空格会被忽略
%       3.1. 为避免歧义,请用花括号({})包裹内容
%   4. 本科生无需填写的项目已被特别标注
%   5. 可以尽情删除本注释

% info 类用于录入个人信息
%   带*号的为对应英文字段
\njusetup[info]{
    title = {第一行标题\\第二行标题\\第三行标题},
    % 中文题目
    % 直接填写标题就是自动换行
    % 可以使用换行控制符(\\)手动指定换行位置
    %
    title* = {My Title in English},
    % 英文题目
    %
    author = {姓名},
    % 作者姓名
    %
    author* = {Ming Xing},
    % 作者英文姓名
    % 一般使用拼音
    %
    keywords = {我,就是,充数的,关键词},
    % 中文关键词列表
    % 使用英文半角逗号(,)分隔
    %
    keywords* = {Dummy,Keywords,Here,{It Is}},
    % 英文关键词
    % 使用英文半角逗号(,)分隔
    %
    grade = {2018},
    % 年级
    %
    student-id = {181850195},
    % 学号或工号
    % 研究生请斟酌大小写字母格式
    % 本模板并不会自动更正大小写
    %
    department = {化学化工学院},
    department* = {School of Chemistry and Chemical Engineering},
    % 院系
    %
    major = {化学},
    major* = {Chemistry},
    % 专业
    %
    % major = {封面专业,摘要专业},
    % 研究生专业型学位可能遇到两处内容不一致的情况
    %
    supervisor = {导师姓名,教授},
    supervisor*= {Professor My Supervisor},
    % 导师全称
    % 使用英文半角逗号(,)分隔中文姓名和职称
    %
    % supervisor-ii = {第二导师姓名,副教授},
    % supervisor-ii* = {Associate professor My Second Supervisor},
    % 第二导师全称
    % 如果确实没有第二导师,不填写即可
    %
    submit-date = {2022-05-20},
    % 提交日期
    % 格式为 yyyy-mm-dd
    % 不填就是编译当天日期
    %
    %
    % 以下均为研究生项
    %
    % degree = {工程硕士},
    % degree* = {Master of Engineering},
    % 覆盖默认学位名称
    %
    field = {物理化学},
    field* = {Physical Chemistry},
    % 研究领域
    %
    chairman = {某某某~教授},
    % 答辩委员会主席
    % 推荐使用波浪号(~)分隔姓名和职称
    %
    reviewer = {
        某某某~教授,
        某某某~教授
    },
    %
    % 答辩委员会成员
    % 一般为四名,使用英文半角逗号(,)分隔
    %
    clc = {O643.12},
    % 中国图书分类号
    %
    udc = {544.4},
    % 国际图书分类号
    %
    secret-level = {公开},
    % 密级
    %
    defend-date = {2022-05-21},
    % 答辩日期
    % 格式为 yyyy-mm-dd
    % 不填就是编译当天日期
    %
    email = {xyz@smail.nju.edu.cn},
    % 电子邮箱地址
    % 只用于出版授权书
    %
    %
    % 以下用于国家图书馆封面
    confer-date = {2022-05-22},
    % 学位授予日期
    %
    bottom-date = {2022-05-23},
    % 封面底部日期
    %
    supervisor-contact = {
        南京大学~
        江苏省南京市栖霞区仙林大道163号
    }
    % 导师联系方式
}

% bib 类用于参考文献设置
\njusetup[bib]{
    % style = numeric|author-year,
    % 参考文献样式
    % 默认为顺序编码制(numeric)
    % 可选著者-出版年制(author-year)
    %
    resource = {njuthesis-sample.bib},
    % 参考文献数据源
    % 需要带扩展名的完整文件名
    % 可使用逗号分隔多个文件
    % 此条等效于 \addbibresource 命令
    %
    % option = {
        % doi    = false,
        % isbn   = false,
        % url    = false,
        % eprint = false,
        % 关闭部分无用文献信息
        %
        % refsection = chapter,
        % 将参考文献表置于每章后
        %
        % gbnamefmt = lowercase
        % 使用仅首字母大写的姓名
    %   }
    % 额外的 biblatex 宏包选项
}

% image 类用于载入外置的图片
\njusetup[image]{
    % path = {{./figure/}{./image/}},
    % 图片搜索路径
    %
    nju-emblem = {nju-emblem},
    nju-name = {nju-name},
    % 校徽和校名图片路径
    % 建议使用 PDF 格式的矢量图
    % 使用外置图片有助于减少编译时间
    % 空置时会自动使用 njuvisual 宏包绘制
    %
    % nju-emblem = {nju-emblem-purple},
    % nju-name = {nju-name-purple},
    % 替换为紫色版本
    % 这个选项只能填写一次
    % 切换时要注释掉上方的黑色版本
}

% abstract 类用于设置摘要样式
\njusetup[abstract]{
    toc-entry = false,
    % 摘要是否显示在目录条目中
    %
    % underline = false,
    % 研究生英文摘要页条目内容是否添加下划线
    %
    % title-style = strict|centered|natural
    % 研究生摘要标题样式,详见手册
}

% 目录自身是否显示在目录条目中
\njusetup{
    tableofcontents/toc-entry = false,
    % 关闭本项相当于同时关闭三个选项
    %
    % listoffigures/toc-entry   = false,
    % listoftables/toc-entry    = false
}

% 为目录中的章标题添加引导线
\njusetup[tableofcontents/dotline]{chapter}

% math 类用于设置数学符号样式,功能详见手册
\njusetup[math]{
    % style              = TeX|ISO|GB,
    % 整体风格,缺省值为国标(GB)
    % 相当于自动设置以下若干项
    %
    % integral           = upright|slanted,
    % integral-limits    = true|false,
    % less-than-or-equal = slanted|horizontal,
    % math-ellipsis      = centered|lower,
    % partial            = upright|italic,
    % real-part          = roman|fraktur,
    % vector             = boldfont|arrow,
    % uppercase-greek    = upright|italic
}

% theorem 类用于设置定理类环境样式,功能详见手册
\njusetup[theorem]{
    % define,
    % 默认创建内置的七种定理环境
    %
    % style         = remark,
    % header-font   = \sffamily \bfseries,
    % body-font     = \normalfont,
    % qed-symbol    = \ensuremath { \male },
    % counter       = section,
    % share-counter = true,
    % type          = {...}
    % 以上设置项在重新调用 theorem/define 后生效
}

% footnote 类用于设置脚注样式,功能详见手册
\njusetup[footnote]{
  % style = pifont|circled,
  % 使用圈码编号
  %
  % hang = false,
  % 不使用悬挂缩进
}

% 页眉页脚内容设置
\njusetup{
  % header/content = {
  %     {OR}{\thepage},{OL}{\rightmark},
  %     {EL}{\thepage},{ER}{\leftmark}
  %   },
  % 页眉设置,详见手册
  % 奇数页页眉:左侧章名,右侧页码
  % 偶数页页眉:左侧页码,右侧节名
  %
  % footer/content = {}
}

% 页眉页脚的字体样式
% \njusetformat{header}{\small\kaishu}
% \njusetformat{footer}{}

% 一些灵活调整
% \njusetname{type}{本科毕业设计}                 % 我做的是毕业设计
% \njusetname{notation}{术语表}                   % 更改符号表名称
% \njusetlength{crulewd}{240pt}                   % 加长封面页下划线
% \njusetformat{tabular}{\zihao{-4}\bfseries}     % 修改表格环境的字号
% \EditInstance{nju}{u/cover/emblem-img}{align=l} % 左对齐的本科生封面校徽


% 自行载入所需宏包
% \usepackage{subcaption} % 嵌套小幅图像,比 subfig 和 subfigure 更新更好
% \usepackage{siunitx} % 标准单位符号
% \usepackage{physics} % 物理百宝箱
% \usepackage[version=4]{mhchem} % 绘制分子式
% \usepackage{listings} % 展示代码
% \usepackage{algorithm,algorithmic} % 展示算法伪代码

% 在导言区随意定制所需命令
% \DeclareMathOperator{\spn}{span}
% \NewDocumentCommand\mathbi{m}{\textbf{\em #1}}

% 开始编写论文
\begin{document}

%---------------------------------------------------------------------
%	封面、摘要、前言和目录
%---------------------------------------------------------------------

% 生成封面页
\maketitle

% 文档默认使用 \flushbottom,即底部平齐
% 效果更好,但可能出现 underfull \vbox 信息
% 如需抑制这些信息,可以反注释以下命令
% \raggedbottom

\begin{abstract}
  中文摘要
\end{abstract}

\begin{abstract*}
  English abstract
\end{abstract*}

% 生成目录
\tableofcontents
% 生成图片清单
% \listoffigures
% 生成表格清单
% \listoftables

%---------------------------------------------------------------------
%	正文部分
%---------------------------------------------------------------------
\mainmatter

% 符号表
% 语法与 description 环境一致
% 两个可选参数依次为说明区域宽度、符号区域宽度
% 带星号的符号表(notation*)不会插入目录
% \begin{notation}[10cm]
%   \item[DFT] 密度泛函理论 (Density functional theory)
%   \item[DMRG] 密度矩阵重正化群 (Density-Matrix Reformation-Group)
% \end{notation}

% 建议将论文内容拆分为多个文件
% 即新建一个 chapters 文件夹
% 把每一章的内容单独放入一个 .tex 文件
% 然后在这里用 \include 导入,例如
%   \chapter{绪论}

\section{研究背景和意义}
分布式协议,如 Paxos和 Raft等,是现代分布式系统的基石。
验证分布式协议的正确性,对保障大规模的数据库系统,云计算系统以及其他分布式系统运行的可靠性和稳定性至关重要。
然而,目前分布式协议愈发复杂,验证分布式协议的正确性并不是一件容易的事情。
在分布式协议验证中,安全属性(safety property)具有至关重要的地位。
如果在运行过程中,系统的每个状态都不违背安全属性,那么我们可以认为这个系统是安全的。
因此,验证分布式协议的正确性,可以转化为验证系统的每个状态都满足安全属性的问题。
对于复杂系统,我们无法简单地采用遍历的方式来验证安全属性是否在每个状态下都成立。
目前的研究方式,大多是寻找一个归纳的不变式,这个不变式经过所有可能的状态转移后仍然能保持其自身的正确性。
并且,这个不变式蕴含着安全属性,即这个不变式成立,则安全属性成立。
这个不变式被称为归纳不变式\cite{inductive}。
寻找到归纳不变式就等同于验证了分布式协议的正确性。\cite{towards}
自动化地生成分布式协议的归纳不变式是验证自动化分布式协议正确性的关键步骤。

\TLA \cite{TLA+}是一个对程序和系统,尤其对并发和分布式的程序和系统进行规约建模的高级语言。
在并发和分布式系统设计和开发过程中,非常容易发生基础性的设计问题,这些问题往往难以被发现。
而\TLA 以及其工具,利用集合论和时态逻辑精确地表达系统的状态和行为,可以帮助开发人员在设计阶段避免这些问题,以及在开发阶段定位问题。

目前,归纳不变式的自动生成技术,大多基于IVy \cite{Ivy} 实现。
然而,IVy的功能相比较\TLA 比较局限,且\TLA 在工业界的应用更加广泛。
当前针对\TLA 规约的自动化归纳不变式生成方法较少,且实现方法比较单一。
我们希望能\TLA 语言上开发出一种新的自动化归纳不变式生成方法,借助机器学习的技术以提高生成效率,为分布式协议的设计和验证提供帮助。

\section{研究问题}
本文使用强化学习的方法,使用 python 语言,基于 \TLA 语言平台,设计了一个自动化归纳不变式生成工具 rlTLA。
本文中的工具对归纳不变式的验证模块使用的 apalache 工具进行验证。
我们还基于 endive 所提供的测试数据集合,对 rlTLA 功能和性能进行了验证和测试。
实验验证了 rlTLA 的有效性和可行性。

\section{国内外研究现状}
目前的归纳不变式生成技术主要是基于 IVy 实现的,科研人员基于 Ivy 的平台设计了诸多归纳不变式自动生成的算法和工具。

从实现理念和思路上,这些工具的大致可以分为两类,一种是基于程序语义(syntax-guided)\cite{syntax}的白盒技术,另一种是基于程序行为的黑盒技术。
近年来,随着 AI-for-SE的发展,一种叫做ICE\cite{ICE}(implication counterexamples)的学习框架流行起来,
它将不变式的证明工作分为了两个部分:学习者和教育者。
依赖随机搜索、决策树\cite{garg2016learning}、强化学习\cite{LIPuS}等技术,许多工作推进了学习者模块的发展。
此外,也有人将新颖的语言大模型引入了不变式生成的工作中\cite{llm}。
白盒技术和黑盒技术的界限并不明确,一些工具其实兼而有之地采取两种技术的优势。

DistAI\cite{DistAI}以及DuoAI\cite{DuoAI}来自同一个研究团队,使用枚举候选不变式的算法进行自动不变式生成。
他们基于已有的小体积的运行数据,在削减过的空间上,在有限的句法空间中通过工具裁剪谓词来生成候选不变式,
他们首先要基于协议的定义,获得一部分的运行数据,即一些协议允许到达的状态。
与此同时,他们还将量词模板进行划分,以减少意义上重复的候选不变式的出现。
然后将获得的状态分配到对应的量词模板中,并考察每个状态下的谓词是否成立。
基于运行数据,他们初步筛选出了一些可能的候选不变式,然后交给工具进行验证,最终得到归纳不变式。
如果运行数据不足以枚举出归纳不变式,他们也会生成更多的运行数据。
总体而言,他们通过小部分的运行数据,削减了搜索空间,减少了验证器的调用次数,提高了不变式生成的效率。

I4\cite{I4}基于有限实例推广进行自动不变式生成。
它首先会根据初始参数,创建一个有限的实例。然后使用 Averroes model checker \cite{goel2019model}生成一个基于小实例的归纳不变式。
如果实例过于复杂,I4 会将实例简化。
之后,I4 会基于已经得到的,小规模实例上的归纳不变式,泛化到一般的归纳不变式。

LIPuS 则在基于语义的基础上,使用了强化学习的框架对搜索空间进行剪枝,并在修剪过后的空间上进行 SMT 求解。
他们首先将程序输入给强化学习框架,让其对总体的不变式模板进行修剪。
然后将修剪之后的模板交给 SMT solver 进行求解。
如果无法求解出来,出现了反例(counterexamples),则将反例交给强化学习框架,让其再次对模板进行修剪。
直到 SMT solver 求解出了不变式,或者强化学习框架无法再修剪出新的模板为止。
使用这种方式可以有效地减少对SMT solver 的调用,从而提高生成归纳不变式的效率。

以上的工作,均是基于 IVy 实现,接受 IVy描述的规约。目前基于 \TLA 的归纳不变式生成工具较少。

IronFleet\cite{IronFleet}和Verdi\cite{Verdi} 是比较早在 \TLA 上实现的分布式系统验证工具。
IronFleet 结合使用细化和简化的方式来加速分布式协议的验证。
而 Verdi 则是使用了一系列的系统转换器。
它先证明比较强约束的模型的正确性,然后通过转换器,将这个模型转换为更弱的模型,再证明这个弱模型的正确性。
事实上,IronFleet 和 Verdi 都离不开人工的加入来验证归纳不变式的正确性,并不是一个完全自动化的归纳不变式生成工具。

endive\cite{endive}是一份基于 \TLA 的自动化归纳不变式生成工具。
endive 需要用户提供原子公式,算法就会根据这些原子公式自动生成可能的归纳不变式,并且从中排除掉那些违反安全属性的不变式。
之后,endive 会选择可以杀死反例最多的不变式,并重复这一过程,直到组合出最终的归纳不变式。
endive 是一种基于 IC3 思想,使用增量搜索,寻找归纳不变式的方法。

\section{本文主要工作}

本文的主要工作包括:
\begin{itemize}
    \item \textbf{预处理:} 对 \TLA  源文件的预处理,获取变量、谓词等,为自动生成归纳不变式模块做准备。
    \item \textbf{系统设计和实现:} 设计了一个基于强化学习的归纳不变式生成工具 rlTLA。其中包括归纳不变式生成模块和归纳不变式验证模块。
其中,生成模块使用强化学习的框架,借助强化学习的优势,优化了归纳不变式的生成效率。
检验模块调用 apalache 工具,对 apalache 返回的结果进行分析,判断归纳不变式的正确性和归纳性质。
    \item \textbf{实验和验证:} 使用 endive 提供的测试数据集合,对 rlTLA 的功能和性能进行验证和测试,证明 rlTLA 的可行性。
\end{itemize}

本项目内容开源于地址:\href{https://github.com/}{github.com}

\section{本文组织结构}
本文的组织结构如下:

第一章主要介绍了项目的研究背景,研究问题,当前国内外在归纳不变式自动生成领域的研究现状,本文的工作内容和组织结构。

第二章介绍预备知识和相关技术。

第三章介绍工具的算法设计和实现,各个模块的功能和行为,以及模块间的交互。

第四章对工具的功能和性能进行验证,并与已有的工具进行对比。

第五章对本文工作进行总结与展望。


%   \include{chapters/environments}

\chapter{引言}

\section{量子计算}

John Preskill 认为我们现在处于“含噪声的中型量子时代”。\cite{preskill2018}
量子位的不稳定性和有限的量子比特数量限制了量子计算机的复杂度和纠错能力。

%---------------------------------------------------------------------
%	参考文献
%---------------------------------------------------------------------

% 生成参考文献页
\printbibliography

%---------------------------------------------------------------------
%	致谢
%---------------------------------------------------------------------

\begin{acknowledgement}
  感谢 \href{https://git.nju.edu.cn/nju-lug/lug-introduction}{LUG@NJU}。
\end{acknowledgement}

%---------------------------------------------------------------------
%	附录部分
%---------------------------------------------------------------------

% 附录部分使用单独的字母序号
% \appendix

% 可以在这里插入补充材料

% 完工
\end{document}
